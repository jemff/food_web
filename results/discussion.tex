\section{Discussion}
\subsection*{What we found} %1 Paragraph
%The game we study is a mean field game (MFG), where the self-interaction is mediated by the other player. Our game has polymorphic-monomorphic equivalency due to the bilinear nature. Mono-Poly is a necessary requirement for models of DVM, since not all copepods/fish behave exactly the same, and we thereby encapsulate minor variations in each fish.

%Short term static driving long term variations
%Strength of numerical approach
%Seasonal variation, driven by population dynamics
%Bounded rationality is the same at long-term, quite different short-term
Our study of a Lotka-Volterra system with optimal behavior in the water column focused on both the emergent behavior and population dynamics. We saw the emergence of populations varying heavily over the long term, from a system where the populations appear static when inspected at short time-scales. The short-term dynamics show a clear distinction between predator and consumer feeding modes, with most predator-prey interactions happening at dusk and dawn. By comparing the results for bounded rationality and unbounded rationality on the long and short time scales, we see two different pictures emerge. The behavior predicted by the two different models are quite different, with a distinct difference in the optimal distributions in the water. Though the behavior is different, the difference averages out over the long time-scale, and the population dynamics for populations with bounded and unbounded rationality are indistinguishable. The results illustrate the strength of our numerical approach to studying population games, and underline the importance of robust algorithms and discretization schemes.


\subsection*{Interpretation of the findings and what they mean} %1-2 paragraphs
%Clear emergence of eeding times, shows importance fo dusk and dawn, maybe possible to average out
%Incorporating populations and threats necessary to specify migration patterns
%The choice of numerical approach is of critical importance, upgrading numerics can lead to new insights
%Unbounded rationality, though unreasonable empirically, is not so important on the large scale but very important on small scale.
Our results illustrate the complicated interplay between population dynamics, seasons, and behavior. Though light levels may be the same at two different times of the year, the migration patterns can differ radically due to the differences in populations. Bounded rationality appears to change the population dynamics imperceptibly compared to the results with perfect rationality, so even though behavior is probably not perfectly rational in reality, the difference between optimal behavior and constant behavior is much greater than between bounded and perfect rationality. Models incorporating the vertical migration as an essential component need to consider the population levels as well as the light levels, essentially incorporating some sort of behavorial element. Incorporating behavior causes non-smooth population dynamics at short time-scales. Initially this seem to indicate that models need to be very fine-grained in time to capture population dynamics accuarely, but the smooth nature of the long-term dynamics gives hope that the same dynamic could be captured with a rougher time-discretization.


\subsection{Compare results to litterature} %2-3 paragraphs
Seasonal variation of the vertical migration due to seasonal changes in nutrient and light levels has been studied in detail in the Arabian sea \citep{wang2014seasonal}, and our results correspond qualitively to these though we are simulating a far more northern latitude. Discounsting seasons, but looking at different environments, the variation of the vertical migration shows a clear dependence on the nutrients and light levels \citep{klevjer2016large}, in vein with what our model predicts.

The shapes of the distributions at both day and night seem to agree better with observations in the model with bounded rationality, \citep{hay1991zooplankton}, lending credence to the idea that copepods are not perfectly rational or subject to turbulence \citep{visser2001observations}.
constant populations \citep{pinti2019trophic,verticalmigration}. Though the populations change dramatically over the long term in our model, the population levels are essentially unchanged on a daily basis. The proposed population-dynamical model passes an essential test, as wildly fluctuating populations in the short term cannot represent the underlying physical reality. In contrast, the near-constant short-term population emerges from the behavorial optimization in the model when comparing to the model with no behavior.


lima2009predators Predators and the breeding bird: behavioral and reproductive flexibility under the risk of predation <- Move to introduction
carranza1999red Red deer females collect on male clumps at mating areas <- Move to introduction
gigerenzer2001rethinking rationality <- Move to introduction
jones1999bounded,
  title={Bounded rationality

cressman2010ideal The ideal free distribution as an evolutionarily stable state in density-dependent population games <- Move to introduction

lasry2007mean mean field games <- Move to introduction

@article{krivan,
  title={On evolutionary stability in predator--Prey models with fast behavioural dynamics},

kvrivan2008ideal The ideal free distribution: a review and synthesis of the game-theoretic perspective
cressman2004ideal Ideal free distributions, evolutionary games, and population dynamics in multiple-species environments

When looking at ecosystems, there are two main approaches to incorporating behavior:
\begin{enumerate}
  \item Assume perfect rationality
  \item Ignore it
\end{enumerate}
The middle way, where animals take rational choices but with a bound on their rationality is typically not included. Imperfect rationality captures that animals might not be perfectly aware of the state of a system, or when distinguishing between almost-equivalent options, choosing the best might require a disprortionate effort.

\subsection{Limitations and generalizability} %1 paragraphs
The model we have developed is ready-made for incorporating more species, for example using it to examine the population dynamics of snapshot models \citep{pinti2019trophic}. Our approach essentially only depends on the linearity of the fitness proxy, so it could be imagined that ontogenics could be incorporated, via. structuring the populations and considering fitness based on the Leslie matrix, enabling a proxy for life-cycle optimization. It is a weakness of the model that the payoff matrices for the predator-prey interactions have to be recalculated at every instant, as this step is very computionally expensives. The shape of the population dynamics gives hope that the population dynamics could be equally well modelled using a modified clearance rate. The Nash equilibrium we find in our model is unique by \citep{verticalmigration}, but in general establishing uniqueness of the Nash equilibrium in a polymatrix game is non-trivial.

\subsection{Further work} %1 paragraph
A weakness of the model is that it does not incorporate a concept of satiation nor ontogenics, and expanding the model to include either of these would be a big improvement. The driver of seasonal variations in migration patterns, can theoretically be tested via. echo-acoustics. The bounded rational distributions we arrive at with our model for zooplankton and forage fish seem to qualitively agree with observations. It would be satisfying to find a general criterion to ensure that a habitat selection game has a unique Nash equilibrium.


\subsection{Conclusion} %1 paragraph
When looking at the population dynamics of a predator-prey system through half a, the seasonal impacts stand out [FF]. Even though our model is simple in nature, it can catch essential features as the spring-bloom, and varying phytoplankton throughout the seaonss.  The long-term patterns are invisible when investigating the short-term fluctuations. As we can see the primary trophic interactions happens in a very short time, the system is fundamentally a slow system driven by a fast underlying dynamic.
We show that it is possible to include bounded rationality in a systematic fashion, allowing it as a tuning parameter in future models. The usual assumption of complete rationality seems to be reasonable based on our results, but care should be taken.
