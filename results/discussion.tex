\section{Discussion}
%\subsection{What we found} %1 Paragraph
%The game we study is a mean field game (MFG), where the self-interaction is mediated by the other player. Our game has polymorphic-monomorphic equivalency due to the bilinear nature. Mono-Poly is a necessary requirement for models of DVM, since not all copepods/fish behave exactly the same, and we thereby encapsulate minor variations in each fish.

%Short term static driving long term variations
%Strength of numerical approach
%Seasonal variation, driven by population dynamics
%Bounded rationality is the same at long-term, quite different short-term
Our study of a Lotka-Volterra system with optimal behavior in the water column focuses on both the emergent behavior and population dynamics. The population dynamics were stabilized dramatically by the introduction of optimal behavior, and a slight seasonal dependence appeared. The short-term dynamics show a clear distinction between predator and consumer feeding modes, \replaced{where prey feed during night while predators feed}{with predator-prey interactions happening} at dusk and dawn. By comparing the results for bounded rationality and unbounded rationality on the long and short time scales, we see two different pictures emerge. The behavior predicted by the two different models is similar, but still visibly different. Though the behavior is different, the difference becomes negligible at long-time scales. The results illustrate the strength of our numerical approach to studying population games, and underline the importance of robust algorithms and discretization schemes.


%\subsection{Interpretation of the findings and what they mean} %1-2 paragraphs
Our results illustrate the interplay between population dynamics, seasons, and behavior. Though light levels may be the same at two different times of the year, the migration patterns can differ radically due to the differences in populations. Bounded rationality appears to change the population dynamics imperceptibly compared to the results with perfect rationality. Behavior is probably not completely rational in reality, but the change from complete rationality to slightly  bounded rationality does not appear important for population dynamics. Assuming perfect rationality appears much more reasonable than full irrationality. At first sight, the intensity of predator-prey interactions at dawn and dusk seem to indicate that models need to be very fine-grained in time to capture population dynamics accurately. Zooming out to the long-term dynamics lends hope that the same dynamic could be captured with a rougher time-discretization.

%Models incorporating the vertical migration as an essential component need to consider the population levels as well as the light levels, essentially incorporating some sort of behavioral element.


%\subsection{Compare results to litterature} %2-3 paragraphs

Our qualitative results on the seasonal variation of the vertical migration tentatively appear to correspond with empirical findings \citep{wang2014seasonal, beaugrand2001geographical, colebrook1979continuous}. This agreement is, it must be emphasized, qualitative in nature. If the model was tuned through empirical data to an ecosystem, perhaps it could be used to forecast seasonal changes in vertical migrations. Large-scale geographical studies of the vertical migration indicate that population levels are a driving factor in the diel vertical migration \citep{klevjer2016large}, not just light. This corresponds to the predictions of our model, and shows the importance of modeling behavior explicitly.
Ocean population dynamics are driven by feeding at dusk and dawn \citep{benoit2014critical}, and our model provides a purely behavioral justification for this phenomenon. At the same time, our results show that the discontinuous feeding patterns gives rise to smooth long-term population dynamics. The population-level results support the usual approach of assuming perfect rationality in population games with interacting populations, though the behavior of each individual is most likely slightly sub-optimal \citep{hurly1999context}. Slight sub-optimality in the vertical distribution is not that important for overall population levels, so at the population level it provides almost no evolutionary benefit.
The model of optimal foraging with bounded rationality we present is testable, contrasting e.g. satisficing models, \citep{nonacs1993satisficing}. Vertical migration patterns of copepod indicate that their distribution in the water column can be closely approximated by the smooth distribution we get from the model with bounded rationality \citep{hay1991zooplankton, visser2001observations}. Our model of bounded rationality passes the first test, but must be compared with more data.
%The exact cause of the slightly suboptimal distribution is not essential in our model.
%Seasonal variation of the vertical migration due to seasonal changes in nutrient and light levels has been studied in detail , and our results  .

%Discounting seasons, but looking at different environments, the variation of the vertical migration shows a clear dependence on the nutrients and light levels , in vein with what our model predicts.

%The shapes of the distributions at both day and night seem to agree better with observations in the model with bounded rationality  lending credence to the idea that copepods are not perfectly rational or subject to turbulence \citep{visser2001observations}.


%\citep{}

%Though the populations change dramatically over the long term in our model, the population levels are essentially unchanged on a daily basis, and our modelling results lend credence to the usual idea of time-scale separation which allows the effect of eg. the vertical migration to be approximated with some coarser fitness proxy, \citep{}.


%s\citep{}

%bounded rationality introduced before, satisficing, not rigorous. our model provides a precise and rigorous way to apply. bounded rationlity most biologically plausible, perfect %ratinoality good enough for coarse models.
%gigerenzer2001rethinking rationality
%jones1999bounded,
%  title={Bounded rationality
%Bounded rationality in C. elegans is explained by circuit-specific normalization in chemosensory pathways
%Hurly, T. A. & Oseen, M. D. Context-dependent, risk-sensitive foraging preferences in wild rufous hummingbirds. Anim. Behav. 58, 59–66 (1999).
%Simon, H. A. A behavioral model of rational choice. Q. J. Econ. 69, 99 (1955).
%Simon, H. A. Rational choice and the structure of the environment. Psychol. Rev. 63, 129–138 (1956).
%Info‐Gap Robust‐Satisficing Model of Foraging Behavior: Do Foragers Optimize or Satisfice?
%Is satisficing an alternative to optimal foraging theory


%Modelling population games as we do presents an example of a sequence of static mean-field games in continuous space, and, or interacting ideal free distributions \citep{cressman2004ideal, cressman2010ideal} if considering the steady-state.
A question that faces our model is how the equilibrium we find at any instant relates to the ideal free distribution. It is unclear whether the distribution we find at any instant satisfies the criteria for being a multi-species ideal free distribution \citep{kvrivan2008ideal}, i.e. stable under best-response dynamics. The Nash equilibrium is unique \citep{verticalmigration}, and the distribution we find corresponds to that in nature. In that sense, it is what would be expected of the ideal free distribution in a multi-species population game \citep{cressman2004ideal}.
Interpreting strategies as distribution in space form the cornerstone of our model. This approach is heavily inspired by that of static mean field games \citep{lasry2007mean, blanchet2016optimal}, rather than classical evolutionary game-theory \cite{hofbauer1998evolutionary}. This focus on the distribution is what gives us the ability to consider continuous habitats, and allows the incorporation of complex spatial features in habitat selection games. Our population model with the vertical migration shows that transient phenomena and continuous fitness gradients \citep{kawecki2004conceptual} are naturally at home in a continuous population game. Incorporating these features allows a more fine-grained biological analysis and greater predictive power in real-world systems than typical finite-patch models \citep{kvrivan2008ideal, sadowski2019predator}. Our focus has largely been on the interplay between temporal and spatial transients in the vertical migration. If one of these is averaged out, e.g. the time-varying habitat, the model can be adapted to find the equilibrium populations and distributions in inhomogeneous habitats.



%kvrivan2008ideal The ideal free distribution: a review and synthesis of the game-theoretic perspective

%cressman2004ideal Ideal free distributions, evolutionary games, and population dynamics in multiple-species environments

%lasry2007mean mean field games
%kvrivan2008ideal The ideal free distribution: a review and synthesis of the game-theoretic perspective
%cressman2004ideal Ideal free distributions, evolutionary games, and population dynamics in multiple-species environments


% allows the incorporation of potentially more , and is a natural way to incorporate closeness between patches.


%The way we find a Nash equilibrium in a habitat game by complementarity methods is not in itself a new development \citep{mariani2016migration}.  The insights from mean-field theory and modern optimization tools were absolutely essential in this work, and reveal the insights that can be gained by creating tighter links between modern mathematical optimization theory and theoretical biology.


%Approaching habitat-selection through a mean-field lens and our use of efficient optimization tools to solve large problems
%Our method fundamentally hinges on having an efficient numerical method to solve the linear complementarity problem, which allows us to consider much finer spatial resolutions than previous ap \citep{mariani2016migration}.
%Our general modeling approach directly provides a model for finding optimal distributions of an arbitrary number of species in a continuous setting, and can be used to find ideal free distributions of stationary habitats in the sense of \citep{cressman2010ideal} by fix-point analysis. The strategy of optimizing at every instant has the advantage of being an evolutionarily stable state \citep{kvrivan2009evolutionary}, so populations where animals follow this strategy cannot be invaded by outsiders.
%OBS OBS OBS OBS OBS OBS OBS
%WRITE ALSO REFERENCE TO IDEAL FREE DISTRIBUTION EMERGING FROM MIGRATION!!!!

%our model is an example of a mean field game, or interacting ideal free distributions in the language of cite cite. the population game we consider does not have explicit density dependence, but could be modified to have such. method goes through, need to take care on the diagonal. example of a static mean field game, in the sense of aumann, others. the strategy is an ess and stable, very good.

%lasry2007mean mean field games
%kvrivan2008ideal The ideal free distribution: a review and synthesis of the game-theoretic perspective
%cressman2004ideal Ideal free distributions, evolutionary games, and population dynamics in multiple-species environments
%cressman2010ideal The ideal free distribution as an evolutionarily stable state in density-dependent population games
%@article{krivan,
%  title={On evolutionary stability in predator--Prey models with fast behavioural dynamics},
%The assumption of the population being idaelly distributioned at every instant hinges on the actual migration dynamics being very fast and the habitat highly interconnected, \citep{cressman2006migration}, so the consideration of migration and population time-scales is very important when using our model.

%It is a weakness of the model that the payoff matrices for the predator-prey interactions have to be recalculated at every instant, as this step is very computationally expensive.
%\subsection{Limitations and generalizability} %1 paragraphs



The model we have developed is ready-made for incorporating more species, and can be used almost without modifications to model the population dynamics of complex aquatic communities \citep{pinti2019trophic}. Viewed more generally, our approach only depends on the linearity of the fitness proxy, so it could be imagined that ontogenics could be incorporated, via. structuring the populations and considering fitness based on the Leslie matrix. Doing so could potentially expand the model to handle life-cycle optimization. Looking at the long-term population dynamics lends hope that the population dynamics could equally well be modeled using a modified clearance rate, and neglecting the daily vertical migration.


%\subsection{Further work} %1 paragraph
A weakness of our model is that it does not incorporate a concept of satiation nor ontogenics. Expanding the model to include either of these would be a big improvement. Performing a multi-year simulation could also reveal interesting patterns, but doing so would complicate the model since it would need to take low-resource adaptations during winter into account. The distributions we arrive at with our model of bounded rationality for zooplankton to qualitatively agree with observations, but tuning the model would show if quantitative predictions are possible. Finally, it would be satisfying to find a general criterion to ensure that a continuous habitat selection game has a unique Nash equilibrium.

%%Physical reality is often continuous. Incorporating continuous structure, patch proximity can be modeled
%Potential price of migration can be incorporated, leading to cost-adjust ideal-free distribution, explaining results of abrams et al *references from 2007 paper, generally also newer papers with migration** with suboptimal habitat-choice.
% OBS OBS OBS

\section{Conclusion} %1 paragraph
Simulating the population dynamics of a predator-prey system through half a year reveal a complex interplay between seasons, behavior and population dynamics. Though our model is simple in nature, it can catch essential features of the seasonal dependence of the diel vertical migration. Though the primary trophic interactions happen abruptly and in a very short time frame when introducing optimal behavior, optimal behavior still serves to stabilize the system.
We show that it is possible to include bounded rationality in a systematic fashion, allowing it as a tuning parameter in future models. The usual assumption of perfect rationality seems to be reasonable for population dynamics, but bounded rationality appears to be better at predicting specific distributions.

%%% Local Variables:
%%% mode: latex
%%% TeX-master: "main"
%%% End:
