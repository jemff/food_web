\section{Discussion}
%\subsection*{What we found} %1 Paragraph
%The game we study is a mean field game (MFG), where the self-interaction is mediated by the other player. Our game has polymorphic-monomorphic equivalency due to the bilinear nature. Mono-Poly is a necessary requirement for models of DVM, since not all copepods/fish behave exactly the same, and we thereby encapsulate minor variations in each fish.

%Short term static driving long term variations
%Strength of numerical approach
%Seasonal variation, driven by population dynamics
%Bounded rationality is the same at long-term, quite different short-term
Our study of a Lotka-Volterra system with optimal behavior in the water column focused on both the emergent behavior and population dynamics. We saw the emergence of populations varying heavily over the long term, from a system where the populations appear static when inspected at short time-scales. The short-term dynamics show a clear distinction between predator and consumer feeding modes, with most predator-prey interactions happening at dusk and dawn. By comparing the results for bounded rationality and unbounded rationality on the long and short time scales, we see two different pictures emerge. The behavior predicted by the two different models are quite different, with a distinct difference in the optimal distributions in the water. Though the behavior is different, the difference averages out over the long time-scale, and the population dynamics for populations with bounded and unbounded rationality are indistinguishable. The results illustrate the strength of our numerical approach to studying population games, and underline the importance of robust algorithms and discretization schemes.


%\subsection*{Interpretation of the findings and what they mean} %1-2 paragraphs
Our results illustrate the complicated interplay between population dynamics, seasons, and behavior. Though light levels may be the same at two different times of the year, the migration patterns can differ radically due to the differences in populations. Bounded rationality appears to change the population dynamics imperceptibly compared to the results with perfect rationality, so even though behavior is probably not perfectly rational in reality, the difference between optimal behavior and constant behavior is much greater than between bounded and perfect rationality, at least for a "small" irrationality as studied here. Models incorporating the vertical migration as an essential component need to consider the population levels as well as the light levels, essentially incorporating some sort of behavioral element. Incorporating behavior causes non-smooth population dynamics at short time-scales, but looking over a full 24-hour cycle we see essentially unchanged population levels, so the model passes an essential realism test. Initially the intensity of predator-prey interactions at dawn and dusk seem to indicate that models need to be very fine-grained in time to capture population dynamics accurately, but the smooth nature of the long-term dynamics gives hope that the same dynamic could be captured with a rougher time-discretization.


%\subsection{Compare results to litterature} %2-3 paragraphs
Seasonal variation of the vertical migration due to seasonal changes in nutrient and light levels has been studied in detail \citep{wang2014seasonal, beaugrand2001geographical, colebrook1979continuous}, and our results correspond qualitatively to these. Discounting seasons, but looking at different environments, the variation of the vertical migration shows a clear dependence on the nutrients and light levels \citep{klevjer2016large}, in vein with what our model predicts. The shapes of the distributions at both day and night seem to agree better with observations in the model with bounded rationality \citep{hay1991zooplankton}, lending credence to the idea that copepods are not perfectly rational or subject to turbulence \citep{visser2001observations}. The model of optimal foraging with bounded rationality we present is testable, contrasting satisficing models,
\citep{nonacs1993satisficing}, and examining the empirical evidence lends credence to our approach. The population-level results lend credence to the usual approach of assuming perfect
rationality in population games with interacting populations, \citep{kvrivan2008ideal}, though the behavior of each individual is most likely slightly sub-optimal
\citep{hurly1999context}.
%\citep{}

%Though the populations change dramatically over the long term in our model, the population levels are essentially unchanged on a daily basis, and our modelling results lend credence to the usual idea of time-scale separation which allows the effect of eg. the vertical migration to be approximated with some coarser fitness proxy, \citep{}.


%s\citep{}

%bounded rationality introduced before, satisficing, not rigorous. our model provides a precise and rigorous way to apply. bounded rationlity most biologically plausible, perfect %ratinoality good enough for coarse models.
%gigerenzer2001rethinking rationality
%jones1999bounded,
%  title={Bounded rationality
%Bounded rationality in C. elegans is explained by circuit-specific normalization in chemosensory pathways
%Hurly, T. A. & Oseen, M. D. Context-dependent, risk-sensitive foraging preferences in wild rufous hummingbirds. Anim. Behav. 58, 59–66 (1999).
%Simon, H. A. A behavioral model of rational choice. Q. J. Econ. 69, 99 (1955).
%Simon, H. A. Rational choice and the structure of the environment. Psychol. Rev. 63, 129–138 (1956).
%Info‐Gap Robust‐Satisficing Model of Foraging Behavior: Do Foragers Optimize or Satisfice?
%Is satisficing an alternative to optimal foraging theory

Modelling population games as we do presents an example of a static mean-field game \citep{lasry2007mean}, or interacting ideal free distributions \citep{cressman2004ideal}. Usually interacting ideal free distributions are only studied in the context of choosing between finitely many patches, \citep{kvrivan2008ideal}. Our general modeling approach directly provides a model for finding ideal free distributions of an arbitrary number of species with a finite number of patches, and discretization path we follow allows the same analysis with a plethora of species, allowing sensitivity tests of habitat choices in entire ecosystems. The ideal free distribution has the advantage of being an evolutionarily stable state \citep{cressman2010ideal, kvrivan2009evolutionary}, so populations where animals follow this strategy cannot be invaded by outsiders.


%our model is an example of a mean field game, or interacting ideal free distributions in the language of cite cite. the population game we consider does not have explicit density dependence, but could be modified to have such. method goes through, need to take care on the diagonal. example of a static mean field game, in the sense of aumann, others. the strategy is an ess and stable, very good.

%lasry2007mean mean field games
%kvrivan2008ideal The ideal free distribution: a review and synthesis of the game-theoretic perspective
%cressman2004ideal Ideal free distributions, evolutionary games, and population dynamics in multiple-species environments
%cressman2010ideal The ideal free distribution as an evolutionarily stable state in density-dependent population games
%@article{krivan,
%  title={On evolutionary stability in predator--Prey models with fast behavioural dynamics},






%\subsection{Limitations and generalizability} %1 paragraphs
The model we have developed is ready-made for incorporating more species, for example using it to examine the population dynamics of snapshot models \citep{pinti2019trophic}. Our approach essentially only depends on the linearity of the fitness proxy, so it could be imagined that ontogenics could be incorporated, via. structuring the populations and considering fitness based on the Leslie matrix, enabling a proxy for life-cycle optimization. It is a weakness of the model that the payoff matrices for the predator-prey interactions have to be recalculated at every instant, as this step is very computationally expensive. Looking at the long-term population dynamics lends hope that the population dynamics could equally well be modeled using a modified clearance rate, and neglecting the daily vertical migration. The Nash equilibrium we find in our model is unique by \citep{verticalmigration}, but in general establishing uniqueness of the Nash equilibrium in a polymatrix game is non-trivial.

%\subsection{Further work} %1 paragraph
A weakness of the model is that it does not incorporate a concept of satiation nor ontogenics, and expanding the model to include either of these would be a big improvement. Performing a multi-year simulation could also reveal interesting patterns, but this would require adding a seasonal dependency to the carrying capacity which we have not done. The bounded rational distributions we arrive at with our model for zooplankton and forage fish seem to qualitatively agree with observations. Finally, it would be satisfying to find a general criterion to ensure that a habitat selection game has a unique Nash equilibrium.
\section*{Conclusion} %1 paragraph
When looking at the population dynamics of a predator-prey system through half a, the seasonal impacts stand out [FF]. Even though our model is simple in nature, it can catch essential features as the spring-bloom, and varying phytoplankton throughout the seaonss.  The long-term patterns are invisible when investigating the short-term fluctuations. As we can see the primary trophic interactions happens in a very short time, the system is fundamentally a slow system driven by a fast underlying dynamic.
We show that it is possible to include bounded rationality in a systematic fashion, allowing it as a tuning parameter in future models. The usual assumption of complete rationality seems to be reasonable based on our results, but care should be taken.

%%% Local Variables:
%%% mode: latex
%%% TeX-master: "main"
%%% End:
