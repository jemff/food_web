\section{Discussion}
%\subsection{What we found} %1 Paragraph
Our study of a Lotka-Volterra system with optimal behavior in the water column focuses on both the emergent behavior and population dynamics. The population dynamics were stabilized dramatically by the introduction of optimal behavior, and a slight seasonal dependence appeared. The short-term dynamics show a clear distinction between predator and consumer feeding modes, where prey feed during night while predators feed at dusk and dawn. By comparing the results for bounded rationality and unbounded rationality on the long and short time scales, we see two different pictures emerge. The behavior predicted by the two different models is similar, but still visibly different. Though the behavior is different, the difference becomes negligible at long-time scales. Of course the effect is a result of the relatively low bounded rationality we have chosen, which was chosen heuristically to smooth the distributions but not too much. The results illustrate the strength of our numerical approach to studying population games, and underline the importance of robust algorithms and discretization schemes.

\todo[inline]{Det med valget af ``bounded rationality'' bør uddybes - det er jo et kontinuum. Diskussionen her får det til at fremstå som om vi har valgt en \emph{a priori} oplagt værdi for niveauet af optimalitet. Så kan vi sige at vi har valgt niveauet af optimalitet sådan at de observerede fordelinger er forskellige fra perfekt optimalitet, mens den emergente populationsdynamik ikke er særligt påvirket?}

%\subsection{Interpretation of the findings and what they mean} %1-2 paragraphs
Our results illustrate the interplay between population dynamics, seasons, and behavior. Though light levels may be the same at two different times of the year, the migration patterns can differ radically due to the differences in \replaced{population abundances}{populations}. Bounded rationality appears to change the population dynamics imperceptibly compared to the results with perfect rationality. Behavior is probably not completely rational in reality, but the change from complete rationality to slightly  bounded rationality does not appear important for population dynamics. Assuming perfect rationality appears much more reasonable than full irrationality. At first sight, the intensity of predator-prey interactions at dawn and dusk seem to indicate that models need to be very fine-grained in time to capture population dynamics accurately. Zooming out to the long-term dynamics lends hope that the same dynamic could be captured with a rougher time-discretization.


%\subsection{Compare results to litterature} %2-3 paragraphs
In terms of the specific case of diel vertical migrations, large-scale geographical studies of the vertical migration indicate that population levels are a driving factor in the diel vertical migration \citep{klevjer2016large}, not just light. This corresponds to the predictions of our model, and shows the importance of modeling behavior explicitly. Our qualitative results on the seasonal variation of the vertical migration tentatively appear to correspond with empirical findings \citep{wang2014seasonal, beaugrand2001geographical, colebrook1979continuous}. This agreement is, it must be emphasized, qualitative in nature. If the model was tuned through empirical data to an ecosystem, perhaps it could be used to forecast seasonal changes in vertical migrations.

At the level of general population games, ocean population dynamics are driven by feeding at dusk and dawn \citep{benoit2014critical}, and our model provides a purely behavioral justification for this phenomenon. At the same time, our results show that the discontinuous feeding patterns gives rise to smooth long-term population dynamics. The population-level results support the usual approach of assuming perfect rationality in population games with interacting populations, though the behavior of each individual is most likely slightly sub-optimal \citep{hurly1999context}. Slight sub-optimality in the vertical distribution is not that important for overall population levels, so at the population level complete rationality provides almost no evolutionary benefit.

Bounded rationality as we introduce it is potentially testable, contrasting e.g. satisficing models, \citep{nonacs1993satisficing}. Empirical studies of copepod vertical migration patterns indicate that their distribution in the water column can be closely approximated by the smooth distribution we get from the model with bounded rationality \citep{hay1991zooplankton, visser2001observations}. Our model of bounded rationality passes the first test, but must be compared with more data.

\todo[inline]{Tydeliggør strukturen i diskussion. Måske vil at markere ``In terms of the specific case of diel vertical migrations,'' vs ``At the level of general population games,'' eller sådan noget.}
%Modelling population games as we do presents an example of a sequence of static mean-field games in continuous space, and, or interacting ideal free distributions \citep{cressman2004ideal, cressman2010ideal} if considering the steady-state.
If we look at the specific distributions we find, it is unclear if they satisfy the criteria for being multi-species ideal free distributions \citep{kvrivan2008ideal}, i.e. stable under best-response dynamics. The Nash equilibrium of the habitat selection game is unique \citep{verticalmigration}, and the distribution we find corresponds to that found in nature. In that sense, it is what would be expected of the ideal free distribution in a multi-species population game \citep{cressman2004ideal}.
\todo[inline]{Skal ovenstående måske ikke bare fjernes? Der er intet indhold...}


\todo{Er det følgende ikke et nyt afsnit?}
Interpreting strategies as distributions in space forms one of the two pillars of our model. This approach is heavily inspired by that of static mean field games \citep{lasry2007mean, blanchet2016optimal}, rather than classical evolutionary game-theory \cite{hofbauer1998evolutionary}. Thinking of the strategy of a population as a distribution rather being at specific locations turns out to be a powerful tool. Thinking in terms of distributions is what allows us to reformulate the game between predators and prey into a continuous polymatrix game, where the Nash equilibrium can theoretically be found. A theoretical reformulation of the population game cannot stand alone, leading to the second pillar of our project: The introduction of efficient numerical methods. A necessity for efficient numerical methods is stating the problem in a form that is easily solvable, and an efficient algorithm to solve that. Using a spectral scheme allows us to only use relatively few points for high precision, \citep{kopriva2009implementing}, giving half the problem statement. Together with a tractable method of solving polymtarix games \citep{miller1991copositive}, we have the full restatement of the problem. We then apply fast algorithms from modern optimization software \citep{Andersson2019, acary2019introduction} to solve the restated problem. This combination is fundamentally what allows us to consider a population game in both continuous time and space.
Overall, our approach is only possible due to an influence from the theory of mathematical optimization. Using insights from mean-field games motivates a change of perspective in how we view habitat selection. An efficient discretization and a good problem formulation allows us to solve the problem efficiently with modern numerical software.




%This focus on the distribution is what gives us the ability to consider continuous habitats, and allows the incorporation of complex spatial features in habitat selection games.





%\todo[inline]{Udmærkede elementer i diskussionen, men den bør struktureres klarere. Det specifikke omkring vertikale vandringer er står fint; .}
Our population model with the vertical migration shows that transient phenomena and continuous fitness gradients \citep{kawecki2004conceptual} are naturally at home in a continuous population game. Incorporating these features allows a more fine-grained biological analysis and greater predictive power in models of real-world systems than typical finite-patch models \citep{kvrivan2008ideal, sadowski2019predator}. Our focus has largely been on the interplay between temporal and spatial transients in the vertical migration. If one of these is averaged out, e.g. the time-varying habitat, the model can be adapted to find the equilibrium populations and distributions in inhomogeneous habitats. We have used our approach for a one-dimensional habitat, but it can readily be used to accommodate a two-dimensional habitat. More generally, our approach only depends on the bilinearity of the fitness proxy and the population-game setup.


%Our model can be used almost without modifications to model the population dynamics of complex aquatic communities \citep{pinti2019trophic}.  As an example, by using a Leslie matrix approach ontogenics could potentially be incorporated in the model. Doing so could potentially expand the model to handle life-cycle optimization.

%\subsection{Further work} %1 paragraph
A weakness of our model is that it does not incorporate satiation or other non-linear effects. Performing a multi-year simulation could also reveal interesting patterns, but doing so would complicate the model since it would need to take low-resource adaptations during winter into account. The distributions we arrive at with our model of bounded rationality for zooplankton qualitatively agree with observations, but to truly test our model of bounded rationality we need to make quantitative predictions and test then. Finally, it would be satisfying to find a general criterion to ensure that a continuous habitat selection game has a unique Nash equilibrium.


\section{Conclusion} %1 paragraph
Simulating the population dynamics of a predator-prey system through half a year reveal a complex interplay between seasons, behavior and population dynamics. Though our model is simple in nature, it can catch essential features of the seasonal dependence of the diel vertical migration. Though the primary trophic interactions happen abruptly and in a very short time frame when introducing optimal behavior, optimal behavior still serves to stabilize the system.
We show that it is possible to include bounded rationality in a systematic fashion, allowing it as a tuning parameter in future models. The usual assumption of perfect rationality seems to be reasonable for population dynamics, but bounded rationality appears to be better at predicting specific distributions.
These results are fundamentally only possible due to one of the major contributions of our approach, namely the introduction of efficient numerical methods for solving continuous population games. Whether to include behavior in a model or not reduces to a question of relevance to the model, not feasibility.
\todo[inline]{Et væsentligt element af bidraget mangler at blive nævnt: at beregningsmæssige metoder (Nash, diskretisering) muliggør at inddrage spil i populationsdynamik.}

%%% Local Variables:
%%% mode: latex
%%% TeX-master: "main"
%%% End:
