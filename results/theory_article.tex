



\begin{definition}[N-player linear game] \label{def:lin_game}
  Let $(X,\mu)$ be a probability space, assuming for simplicity that $\mu$ is either discrete or  continuous. Assume that we have $N$ players with utility functions $U_i$ and strategies $\phi_i$. We assume that $\phi_i$ is a probability distribution, as well as lying in $L^2(X,\mu)$. Denote the space of feasible $\phi_i$ as $K$. The pairwise interactions are defined by a family of bounded linear operators $A_{ij}: L^1(X)\to L^2(X)$, ie. the payoff of player $i$ playing strategy $\phi_i$ against player $j$ with strategy $\phi_j$ is $\gamma_{ij}=\ip{\phi_i}{A_{ij} \phi_j}$. The total expected payoff $U_i(\phi^{-i})$ of player $i$ is given by
  \begin{equation}
    \ip{\phi_i}{\sum_{k=1}^N A_ij \phi_j}
  \end{equation}
\end{definition}
\begin{remark}
  Setting $X$ to be a finite set of points equipped with the normalized counting measure in \Cref{def:lin_game}, we recover the notion of a polymatrix game.
\end{remark}
\begin{lemma}
  The space $K\subset L^2(X)$ of feasible $\phi$ is closed and convex.
\end{lemma}
\begin{proof}
  Consider the functional $f$ given by $f(\phi)=\ip{1}{\phi}$, and let $L^2(X)_+$ be the subspace of a.e. positive square-integrable functions. Then $K$ can be written as $f^{-1}(\{1\}) \cap L^2(X)_+$, an intersection of two closed and convex sets, showing the desired.
\end{proof}

\begin{definition}
  Let $X$ be a measure space. Leting $A_{ij}$ be a finite collection of bounded operators $A_{ij}: L^1(X) \to L^2(X)$. Define the total payoff operator $M:K^N \to K^N$ as:
  \begin{equation}
    M =
    \begin{bmatrix}
        0 & A_{1,2} & A_{1,3} &\dots & A_{1,N} \\
        A_{2,1} & 0 & A_{2,3} &\dots & A_{2,N} \\
        \vdots & \vdots & \vdots & \vdots & \vdots \\
        A_{N,1} & A_{N,2} & \dots & \dots & 0
    \end{bmatrix}
  \end{equation}
  The operator $M$ can be assumed negative on $K$ without loss of generality, by subtracting the maximal supremum of the $A_{ij}$ on $K$, which exists as they are bounded from $L^1(X)$.
\end{definition}
%\begin{remark}
  %If we consider an finite amount of players, and require that the family be uniformly bounded, the construction goes through as well.
%\end{remark}
\begin{remark}
  We can weaken the assumption of the $A_{ij}$ being bounded on $L^1(X)$ to $L^2(X)$ if we consider $K\cap B_n$, where $B_n$ is the unit ball in $L^2(X)$ of radius $n$.
\end{remark}


The requirements for a family $\Phi^*=(\phi_i^*)_{i=1}^N \in K^N$ to constitute a Nash equilibrium is: There is no other family $(\psi_i)_{i=1}^N$, such that
\begin{equation}
    \ip{\phi_i^*}{\sum_{k=1}^N A_ij \phi_j^*} < \ip{\psi_i}{\sum_{k=1}^N A_ij \phi_j^*}
\end{equation}
for all $i\in \{1,\dots,N\}$.
By linearity, this is equivalent to the condition that there is no point $\Phi \in K^n$ such that
\begin{equation}
  \ip{\Phi^*-\Phi}{M \Phi^*} < 0
\end{equation}
For $\Phi, \Phi^* \in K^n$, define the slack variable $d = \Phi - \Phi^*$.  If $\phi_i^* = 0$ a.s., then $d_i \geq 0$ a.s., and $\int d_i d\mu(x) \geq 0$. Thus a vector $\Phi^* \in K^n$ is an equilibrium if and only if there is no $d \in L^2(X)^N$ such that
\begin{equation}
  &\ip{-d}{R \Phi} < 0 \\
  -\int d_i \mu(x) \leq 0 \\
  \phi_i = 0 \Rightarrow -d_i \leq 0
\end{equation}
The generalized version of Farkas Theorem states: Given locally convex spaces $X$ and $Y$ and a strongly continuous linear map $A$, if $A(S)$ is strongly closed the following conditions are equivalent:
\begin{align}
  Ax = b \text{ has a solution } x\in S
  A^* y^* \in S^* \Rightarrow \ip{y^*}{b}
\end{align}
This allows us to formulate our problem in the terms of a complementarity problem:
\begin{equation}
  &Mz+q=w \\
  &\ip{w}{z} = 0
  z \in K, w\in K
  \label{eq:complementarity_problem}
\end{equation}


\begin{theorem}
  We can state the existence of a Nash equilibrium in two cases:
  \begin{enumerate}
    \item
    If $A_{ij}:L^1(X) \to L^2(X)$, then for $K = L^2(X)_+$, \Cref{eq:complementarity_problem} has a solution, giving a Nash equilibrium of \Cref{def:lin_game} if all $A_{ij}$ satisfy: (Insert criteria from theorem 3)
    \item
      If we consider the weakly compact convex set given by $L^2(X)_+ \cap B_n$, the game in \Cref{def:lin_game} has a Nash equilibrium, and it is given by \Cref{eq:complementarity_problem}
  \end{enumerate}
\end{theorem}
\begin{example}

\end{example}


%Assume without loss of generality that $\ip{\phi_i}{A_{ij} \phi_j}$ is negative for any probability measures $\phi_i,\phi_j$, reversing the inequality.
%\begin{equation}
  %\label{eq:inequality}
  %  \ip{\phi_i^*}{\sum_{k=1}^N A_ij \phi_j^*} \leq \ip{\psi}{\sum_{j=1}^N A_ij \phi_j^*}
%\end{equation}
%For a fixed family of strategies, define $v_i = \ip{\phi_i^*}{\sum_{j=1}^N A_ij \phi_j^*} $. For any measurable subset $C$ of non-zero measure define a probability measure $\psi_c = c 1_C$. Since \Cref{eq:inequality} holds for all $\psi_c$, we can conclude that we have the inequality
%\begin{align}
%  \sum_{i=1}^N A_{ij} \phi^*_j \geq v_i 1_X \text{ a.s.}
%\end{align}
%This implies that
%\begin{equation}
%  \ip{\phi_i^*}{\sum_{j=1}^N A_ij\phi_j^* - v_i 1_X } = 0
%\end{equation}

@article{craven1977generalizations,
  title={Generalizations of Farkas’ theorem},
  author={Craven, Bruce Desmond and Koliha, Jaromir Joseph},
  journal={SIAM Journal on Mathematical Analysis},
  volume={8},
  number={6},
  pages={983--997},
  year={1977},
  publisher={SIAM}
}
(5)
@article{glicksberg1952further,
  title={A further generalization of the Kakutani fixed point theorem, with application to Nash equilibrium points},
  author={Glicksberg, Irving L},
  journal={Proceedings of the American Mathematical Society},
  volume={3},
  number={1},
  pages={170--174},
  year={1952},
  publisher={JSTOR}
}
(2)


@article{goeleven1993solvability,
  title={On the solvability of noncoercive linear variational inequalities in separable Hilbert spaces},
  author={Goeleven, D},
  journal={Journal of Optimization Theory and Applications},
  volume={79},
  number={3},
  pages={493--511},
  year={1993},
  publisher={Springer}
}
Theorem 3.3 and Remark 3.2 and Prop 3.1

ScriptvWlient for inspiration details missing
