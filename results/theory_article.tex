
The following theorem is an adaption of \citep[Theorem 3.3]{} to our setting, whereby we show that the requirement of lower semicontinuity can be replace by the condition that the convex set is instead a convex cone. We largely follow their notational conventions.
\begin{theorem}[Variational inequality formulation]
\label{thm:lcp_hilbert}
Let $K$ be a non-empty closed convex cone in a real separable Hilbert space H. Let $T:H \to H$ be a bounded linear operator, and $q\in H$. Assume:
\begin{enumerate}
	\item For all $x \in K$, $\ip{x}{Tx} \geq 0$, and $\ip{x}{Tx} = 0$ implies that $\ip{x}{(T+T^*)x} = 0$.
 	\item $T$ satisfies: For $(x_n)$ a sequence in $K$ of unit norm, such that $\liminf \ip{x_n}{T x_n} = 0$, there is a subsequence $x_{n_k}$ converging to $x^*$ different from 0.
\end{enumerate}
Then for every $q$ in $W = \int(\{w \in X : \ip{w}{x} \geq 0\, \quad x  \in \int(ker(T+T^*) \cap K})$, the problem
\begin{equation}
	\ip{v-\overline{x}}{T\overline{x} + q} \geq 0, \quad \text{ for all } v\in K
\end{equation}
has a solution $\overline{x} \in K$.
\end{theorem}
\begin{proof}
	Assume $q \in W - T(K)$, then there exists $x_0$ such that
	\begin{equation}
		\label{eq:var_ineq}
		\ip{y}{q+Tx_0} > 0, \forall y \in \int(ker(T+T^*) \cap K
	\end{equation}
	Let $(\psi_n)_{n\in \N}$ be a sequence which is dense in $K$, with $\psi_0 = x_0$ and and define $X_n$ as the closure of the space spanned by the first $n$ elements, with corresponding projection $p_n: X \to X_n$, converging strongly to the identity on $X_{\infty} = \bigcup_{n \in \N} X_n$. Definining $K_n = X_n \cap K$, we have a natural system of inclusions of $K_n$ in $K_{n+1}$, defining $K^\infty = \bigcup_{\N} K_n \subset X_\infty$, we have $K^\infty = K$, and $p_n \to 1_K$ strongly. We can apply this system of projections to attain a system of finite-dimensional solutions to \Cref{eq:var_ineq}, by considering
\begin{equation}
	\ip{p_n y}{q+(p_nTp_n) x_0} \geq 0
\end{equation}
by \Cref{thm:goeleven2}, each of the variational inequalities has a solution $u_n$. It remains to show that the sequence $u_n$ is uniformly bounded. Assume for contradiction that $u_n$ is unbounded, and consider the quantities $x_n = u_n/\norm{u_n}$ and $q_n = q / \norm{u_n}$. The sequence $x_n$ is clearly bounded, and since $u_n \in K_n$ for each $n$ and $K$ is assumed to be a (convex) cone, then $x_n \in K_n$ for each $n$ as well, therefore the weak limit of a subsequence $x_{n_k}$, $x^*$, is in $K^\infty$ and different from 0. Fix $v_0 \in K$, then there is an $n$ such that $v_0 \in K_n,\quad n \geq m$. For every $n \geq m$ we have the inequality
\begin{equation}
	\ip{-v_0/\norm{u_n}+x_n}{Tx_n+q_n} \leq 0
\end{equation}
By taking the weak limit $x_n \to x^* \in K^\infty$, we arrive at $\ip{x^*}{Tx^*} \leq 0$, which implies that $\ip{x^*}{Tx^*}=0$, so $T^*x^* = - Tx^*$. Thus by assumption (2) we see that we can choose a subsequence such that the limit $x^*$ is different from 0.  By assumption 2 in goeleven *not included here* (Check where we use this, I think it is immediate, it is about the non-empty interior of $K^+(T) =\{h\in H : \ip{h}{y} \geq 0 : \forall y \in ker(T+T^*) \cap K\} $), for every $q$ in $K$ there exists an $x_0$ such that
\begin{align*}
	\ip{Tx_0+q}{v} > 0, \quad \forall v \in ker(T+T^*)\cap K
\end{align*}
In addition, by positivity of $T$, we can write up the inequality
\begin{align*}
	\ip{x_0}{Tx_n + q} - \ip{x_n}{q} \geq 0
\end{align*}
So $\ip{x_0}{Tx_n+q} \geq \ip{x_n}{q}$, taking limits gives $\ip{x_0}{Tx^*+q} \geq \ip{x^*}{q}$
Which is a contradiction to the previous inequality. Therefore the sequence $u_n$ must be uniformly bounded, and we can without loss of generality assume that $u_n$ is weakly convergent to $u^*$, else we could pass to a subsequence.
Going back to the finite-dimensional case, we have the sequence of inequalities for any $x \in K$
\begin{equation}
	\ip{x - u_n}{q+Tu_n} \geq 0
\end{equation}
Splitting up, we arrive at
\begin{align*}
	\ip{x-u_n}{q}+\ip{x}{Tu_n}-\ip{u_n}{Tu_n} \geq 0
	\limsup_{n}(\ip{x-u_n}{q}+\ip{x}{Tu_n}) \geq \limsup_{n}\ip{u_n}{Tu_n} \geq 0
	\lim_{n} \ip{x-u_n}{q} + \lim_{n} \ip{x}{Tu_n} \geq \limsup_{n} \ip{u_n}{T u_n} \geq \ip{u^*}{Tu^*}
	\ip{x-u^*}{q} + \ip{x}{Tu_n^*} \geq \ip{u^*}{Tu^*}
\end{align*}
showing that $\ip{x-u^*}{q}+\ip{x}{Tu^*}-\ip{u_n}{Tu^*} \geq 0$, thus there exists a solution to the problem. The solution set is bounded by an analogous argument to the boundedness of $u_n$,, and clearly closed, hence weakly compact.
\end{proof}
\begin{corollary}[Linear complementarity problem]
  \label{cor:lcp_formulation}
  Let $H$ be a separable real Hilbert space, and define the closed convex cone $K=H^+$ consisting of positive elements. Assume $T$ satisfies the assumptions of \Cref{thm:lcp_hilbert}. Then the linear complementarity problem
  \begin{align*}
    &\ip{x}{Tx+q} = 0
    &Tx+q \in K, ~x \in K
  \end{align*}
  has a solution for $q \in H$ s.t. $\ip{q}{x} > 0$ where $x\in \ker(T+T^*) \cap K - T(K)$ .
\end{corollary}
\begin{proof}
  Set $v=0$ in \Cref{thm:lcp_hilbert}, then $\ip{\overline{x}}{T\overline{x} + q} \geq 0$. Inserting $v=2\overline{x}$, we arrive at
  $\ip{\overline{x}}{T\overline{x}+q}\leq 0$, so $\ip{x}{T\overline{x}+q} = 0$.
\end{proof}

\begin{definition}[N-player linear game] \label{def:lin_game}
  Let $(X,\mu)$ be a probability space, assuming for simplicity that $\mu$ is either discrete or  continuous. Assume that we have $N$ players with utility functions $U_i$ and strategies $\phi_i$. We assume that $\phi_i$ is a probability distribution, as well as lying in $L^2(X,\mu)$. Denote the space of feasible $\phi_i$ as $K$. The pairwise interactions are defined by a family of bounded linear operators $A_{ij}: L^2(X)\to L^2(X)$, ie. the payoff of player $i$ playing strategy $\phi_i$ against player $j$ with strategy $\phi_j$ is $\gamma_{ij}=\ip{\phi_i}{A_{ij} \phi_j}$. The total expected payoff $U_i(\phi^{-i})$ of player $i$ is given by
  \begin{equation}
    \ip{\phi_i}{\sum_{k=1}^N A_ij \phi_j}
  \end{equation}
\end{definition}
\begin{remark}
  Setting $X$ to be a finite set of points equipped with the normalized counting measure in \Cref{def:lin_game}, we recover the notion of a polymatrix game.
\end{remark}
\begin{lemma}
  The space $K\subset L^2(X)$ of feasible $\phi$ is closed and convex.
\end{lemma}
\begin{proof}
  Consider the functional $f$ given by $f(\phi)=\ip{1}{\phi}$, and let $L^2(X)_+$ be the subspace of a.e. positive square-integrable functions. Then $K$ can be written as $f^{-1}(\{1\}) \cap L^2(X)_+$, an intersection of two closed and convex sets, showing the desired.
\end{proof}

\begin{definition}
  \label{def:total_payoff}
  Let $(X,\mu)$ be a probability space, $K$ a convex subset of $L^2(X)$, and $A_{ij}$ a finite collection of bounded operators $A_{ij}: K \to L^2(X)$. Define the total payoff operator $M:K^N \to K^N$ as:
  \begin{equation}
    M =
    \begin{bmatrix}
        0 & A_{1,2} & A_{1,3} &\dots & A_{1,N} \\
        A_{2,1} & 0 & A_{2,3} &\dots & A_{2,N} \\
        \vdots & \vdots & \vdots & \vdots & \vdots \\
        A_{N,1} & A_{N,2} & \dots & \dots & 0
    \end{bmatrix}
  \end{equation}
  %The operator $M$ can be assumed negative on $K$ without loss of generality, by subtracting the maximal supremum of the $A_{ij}$ on $K$, which exists as they are bounded from $L^1(X)$.
\end{definition}
\begin{remark}
  We can weaken the assumption of the $A_{ij}$ being bounded on $L^1(X)$ to $L^2(X)$ if we consider $K\cap B_n$, where $B_n$ is the unit ball in $L^2(X)$ of radius $n$.
\end{remark}
We recall the generalized version of Farkas Lemma for future use:
\begin{lemma}
  \label{lem:farkas_lemma}
Given locally convex spaces $X$ and $Y$ and a strongly continuous linear map $A:X\to Y$, if $A(S)$ is strongly closed the following conditions are equivalent:
  \begin{align}
    Ax = b \text{ has a solution } x\in S
    A^* y^* \in S^* \Rightarrow \ip{y^*}{b} \geq 0
  \end{align}
\end{lemma}

\begin{theorem} \label{thm:nash_eq}
  Let $(X,\mu)$ be a probability space, and assume that we have an $N$-player game in the sense of \Cref{def:lin_game}. Then the Nash equilibrium of \Cref{def:lin_game} can be found by solving the complementarity problem
  \begin{equation}
    &Mz+q=w \\
    &\ip{w}{z} = 0
    z \in K, w\in K
    \label{eq:complementarity_problem}
  \end{equation}
  for two different possibilities of $K$.
  \begin{enumerate}
    \item
      If all the operators $-A_{ij}$ satisfy the assumption \Cref{thm:lcp_formulation}
      Then the problem of finding a Nash equilibrium on $K = L^2(X)_+$ can be written as \Cref{eq:complementarity_problem}, giving a Nash equilibrium of \Cref{def:lin_game}.
    \item
      If we consider the weakly compact convex set given by $L^2(X)_+ \cap B_n$, the game in \Cref{def:lin_game} has a Nash equilibrium, and it is given by \Cref{eq:complementarity_problem}
  \end{enumerate}
\end{theorem}
\begin{proof}
  \begin{enumerate}
    \item
    In case the $A_{ij}$ are compact, consider $A_{ij}-\lambda$ instead. This does not change the Nash equilibrium of the game, since all payoffs are changed by the same amount. In this way, we are guaranteed that all $-A_{ij}$ are semi-coercive.
    The requirements for a family $\Phi^*=(\phi_i^*)_{i=1}^N \in K^N$ to constitute a Nash equilibrium is: There is no other family $(\psi_i)_{i=1}^N$, such that
    \begin{equation}
        \ip{\phi_i^*}{\sum_{k=1}^N A_ij \phi_j^*} < \ip{\psi_i}{\sum_{k=1}^N A_ij \phi_j^*}
    \end{equation}
    for all $i\in \{1,\dots,N\}$.
    By linearity, this is equivalent to the condition that there is no point $\Phi \in K^n$ such that
    \begin{equation}
      \ip{\Phi^*-\Phi}{M \Phi^*} < 0
    \end{equation}
    Define the total payoff operator as $R$ from \Cref{def:total_payoff}. Since $-R$ is assumed positive-plus, we can slack the requirement that $\int \phi_i d\mu(x)=1$ to $\int \phi_i d\mu(x) \geq 1$, as a greater variation decreases the payoff.

    For $\Phi, \Phi^* \in K^n$, define the slack variable $d = \Phi - \Phi^*$. If $\phi_i^* = 0$ a.s., then $d_i \geq 0$ a.s., and $\int d_i d\mu(x) \geq 0$.

    Thus a vector $\Phi^* \in K^n$ is an equilibrium if and only if there is no $d \in L^2(X)^N$ such that
    \begin{equation}
      &\ip{-d}{M \Phi} < 0 \\
      \ip{\phi_i}{1} = 1 \Rightarrow -\int d_i \mu(x) \leq 0 \\
      \phi_i = 0 a.s. \text{ on } A\subset X \Rightarrow -d_i \leq 0  \text{ on } A\subset X
    \end{equation}
    Defining the operator $C:\osum_{i=1}^N \ip{1}{\cdot}: H^n \to \R^n$.
    Writing this with block-matrix operators, the existence of a Nash equilibrium is predicated on \Cref{eq:alternative_eq} having no solution
    \begin{equation}
      \label{eq:alternative_eq}
        \ip{-d}{R\Phi^*}<0 \\
        \begin{pmatrix}
          A & -\ip{1}{\dot} - \ip{\dot}{A\Phi} \\
          -I_H^n & \ip{\dot}{\Phi^*}
        \end{pmatrix}
        \begin{pmatrix}
          d \\
          L
        \end{pmatrix}
        \leq 0
    \end{equation}
    We can introduce the block-matrix operators:
    \begin{align}
        B = \begin{bmatrix} A & -\ip{1}{\dot} \\ -I^n_H & \Phi^* \end{bmatrix}
        b = \begin{pmatrix} M\Phi^* \\ 0 \end{pmatrix}
    \end{align}
    By semi-coercitivity, the sets $A_{ij}(K)$ is closed so by \Cref{lem:farkas_lemma}, we see that \Cref{eq:alternative_eq} has no solution if and only if the system:
    \begin{equation}
      \ip{B^*}{w} = b, \quad w \in L^2_+
    \end{equation}
    has a solution.
    This allows us to formulate our problem in the terms of a complementarity problem:
    \begin{equation}
      &Mz+q=w \\
      &\ip{w}{z} = 0
      z \in K, w\in K
    \end{equation}
    which has a solution by \Cref{cor:lcp_formulation}, giving the Nash equilibrium.
  \end{enumerate}
\end{proof}
\begin{corollary}
  If a problem is formulated in terms of \Cref{def:lin_game} and has a unique Nash equilibrium, this can be determined by considering case (2) in \Cref{thm:nash_eq} and increasing $n$ monotonically.
\end{corollary}
\begin{corollary}
  If the operators in case(1) of \Cref{thm:nash_eq} are bounded viewed as operators from $L^1(X)$ to $L^2(X)$, then we can dispense with the requirement of copositivity plus.
\end{corollary}
\begin{corollary}
  and Remark 3.2 and Prop 3.1
\end{corollary}
\begin{example}

\end{example}


%Assume without loss of generality that $\ip{\phi_i}{A_{ij} \phi_j}$ is negative for any probability measures $\phi_i,\phi_j$, reversing the inequality.
%\begin{equation}
  %\label{eq:inequality}
  %  \ip{\phi_i^*}{\sum_{k=1}^N A_ij \phi_j^*} \leq \ip{\psi}{\sum_{j=1}^N A_ij \phi_j^*}
%\end{equation}
%For a fixed family of strategies, define $v_i = \ip{\phi_i^*}{\sum_{j=1}^N A_ij \phi_j^*} $. For any measurable subset $C$ of non-zero measure define a probability measure $\psi_c = c 1_C$. Since \Cref{eq:inequality} holds for all $\psi_c$, we can conclude that we have the inequality
%\begin{align}
%  \sum_{i=1}^N A_{ij} \phi^*_j \geq v_i 1_X \text{ a.s.}
%\end{align}
%This implies that
%\begin{equation}
%  \ip{\phi_i^*}{\sum_{j=1}^N A_ij\phi_j^* - v_i 1_X } = 0
%\end{equation}

@article{craven1977generalizations,
  title={Generalizations of Farkas’ theorem},
  author={Craven, Bruce Desmond and Koliha, Jaromir Joseph},
  journal={SIAM Journal on Mathematical Analysis},
  volume={8},
  number={6},
  pages={983--997},
  year={1977},
  publisher={SIAM}
}
(5)
@article{glicksberg1952further,
  title={A further generalization of the Kakutani fixed point theorem, with application to Nash equilibrium points},
  author={Glicksberg, Irving L},
  journal={Proceedings of the American Mathematical Society},
  volume={3},
  number={1},
  pages={170--174},
  year={1952},
  publisher={JSTOR}
}
(2)


@article{goeleven1993solvability,
  title={On the solvability of noncoercive linear variational inequalities in separable Hilbert spaces},
  author={Goeleven, D},
  journal={Journal of Optimization Theory and Applications},
  volume={79},
  number={3},
  pages={493--511},
  year={1993},
  publisher={Springer}
}
Theorem 3.3 and Remark 3.2 and Prop 3.1

ScriptvWlient for inspiration details missing
