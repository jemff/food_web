\begin{abstract}

  Population dynamics in the ocean are generally modelled without taking behavior into account. This in spite of the largest daily feeding times for predators, namely at dawn and dusk, being driven by behavior. The daily pattern stems from the Diel Vertical Migration (DVM). This is usually explained by prey avoiding visual predators, and visual predators seeking to find prey. We develop a game-theoretical model of predator-prey interactions in continuous time and space, finding the Nash equilibrium at every instant. Our approach allows a unified model for the slow time-scale of population dynamics, and the fast time-scale of the vertical migration, under seasonal changes.
  On the behaviorial time-scale, we see the emergence of a deep scattering layer from the game dynamics. On the longer time-scale of population dynamics, the introduction of optimal behavior has a strong stabilizing, compared to the model without optimal behavior. In a changing seasonal environment, we observe a change in daily migration patterns throughout the seasons, driven by changes in both population and light levels.
\end{abstract}
\section{Introduction}

The diel vertical migration, (DVM), is the largest migration of organisms on earth, \citep{}, dwarfing the great migrations on the african savannah and bird migrations, \citep{}. The diel vertical migration is the daily migration of millions of animals from the upper layers of the ocean at night to the deeper, darker, layers at daytime.
