\begin{abstract}

  Population dynamics in the ocean are generally modelled without taking behavior into account. This in spite of the largest daily feeding times for predators, namely at dawn and dusk, being driven by behavior. The daily pattern stems from the Diel Vertical Migration (DVM). This is usually explained by prey avoiding visual predators, and visual predators seeking to find prey. We develop a game-theoretical model of predator-prey interactions in continuous time and space, finding the Nash equilibrium at every instant. By unifying results for the general resolution of polymatrix games, and a spectral discretization scheme, we can resolve the spatially continuous game nearly instantaneously. Our approach allows a unified model for the slow time-scale of population dynamics, and the fast time-scale of the vertical migration, under seasonal changes.
  On the behaviorial time-scale, we see the emergence of a deep scattering layer from the game dynamics. On the longer time-scale of population dynamics, the introduction of optimal behavior has a strong stabilizing, compared to the model without optimal behavior. In a changing seasonal environment, we observe a change in daily migration patterns throughout the seasons, driven by changes in both population and light levels. The framework we propose can easily be adapted to population games in inhomogenous terrestrial environments, and more complex food-webs.
\end{abstract}
\section{Introduction}

The diel vertical migration, (DVM), is the largest migration of organisms on earth, \citep{}, dwarfing the great migrations on the african savannah and annual bird migrations, \citep{}. During the day of billions of small fish and zoo-plankton migrate from the upper layers of the ocean to the deeper, darker layers. With sun down, they begin migrating upwards again, staying in the mixed layer at night to feed. At day, the migration has been measured world-wide as the deep scattering layer, \citep{}. The migration acts as a driver of ocean population dynamics, with a majority of predator-prey interactions taking place at dusk and dawn in the mixed layer, \citep{}.


Though understanding the DVM is of central importance in understanding ocean dynamics and carbon flux, \citep{}, the cause of the migration is not entirely clear. Performing the migration must confer some advantage, measured in terms of increased fitness. A dominating theory is that the DVM is driven by the attempt to avoid risk from visual predation, and as such is driven by the abundance of light, \citep{}.


Though the DVM acts as a driver for ocean population dynamics, most models incorporating behavior do not incorporate population dynamics. When models include population dynamics, it is by discretizing the ocean into two or three zones and losing the continuous nature of the water column. Conversely, models of the DVM with a continuous ocean do not work with population dynamics. 



Our approach provides a unified framework for examining the population and behavorial time-scales.



Understanding advanced interest in undersntading the emergent patterns from the vertical game between predator and prey
 %Iwasawa et al.

%Reference Jerome, UHTH



%
Our method of solving the propose a novel method to solve continuous multi-population games, applying it to the
