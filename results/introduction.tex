\begin{abstract}

  Population dynamics in the ocean are generally modelled without taking behavior into account. This in spite of the largest daily feeding times for predators, namely at dawn and dusk, being driven by behavior. The daily pattern stems from the Diel Vertical Migration (DVM). This is usually explained by prey avoiding visual predators, and visual predators seeking to find prey. We develop a game-theoretical model of predator-prey interactions in continuous time and space, finding the Nash equilibrium at every instant. By unifying results for the general resolution of polymatrix games, and a spectral discretization scheme, we can resolve the spatially continuous game nearly instantaneously. Our approach allows a unified model for the slow time-scale of population dynamics, and the fast time-scale of the vertical migration, under seasonal changes.
  On the behaviorial time-scale, we see the emergence of a deep scattering layer from the game dynamics. On the longer time-scale of population dynamics, the introduction of optimal behavior has a strong stabilizing, compared to the model without optimal behavior. In a changing seasonal environment, we observe a change in daily migration patterns throughout the seasons, driven by changes in both population and light levels. The framework we propose can easily be adapted to population games in inhomogenous terrestrial environments, and more complex food-webs.
\end{abstract}
\section{Introduction}

The diel vertical migration (DVM) is a defining structure of oceanic life, \citep{olivar2017mesopelagic}. During the day billions of small fish and zoo-plankton migrate from the upper layers of the ocean to the deeper, darker layers, which can be directly measured as the deep scattering layer, \citep{sutton2013vertical, wang2014seasonal}. At dusk, the small fish and zoo-plankton migrating upwards, staying in the mixed layer at night to feed.
The enornomous vertical movement of copepod and pelagic populations form a vital part of the biological carbon pump, \citep{hansen2016carbon, ducklow2001upper}.

The mechanisms driving the DVM are not entirely understood, and improving this understanding is a prerequisite for understanding the composition of marine food-webs and mechanisms of carbon transfer. Performing the migration must confer some advantage, measured in terms of increased fitness. A dominating theory is that the DVM is driven by the attempt of small forage fish copepods shifting to avoid risk from visual predation, and as such is driven by the abundance of light, \citep{nilsson2003vertical}.

Predators naturally follow their food source, confirmed \citep{sims2005habitat} with zoo-plankton feeding baskin sharks following the plankton. The distribution of predators depends on the distribution of prey, and vice versa. In a sense, the predators and prey are playing a game of hide and seek across the water column.


Viewing the DVM as an emergent phenomenom from behavorial optimization was pioneered by \citep{iwasa1982vertical}. Qualitive characteristics of the DVM emerged purely from behavior, such as the emergence of the deep scattering layer, and a mixing of predators and prey in the mixed layer at night. This was accomplished by discretizing the water column in two zones, an upper layer and a dark lower layer. The day-night cycle into a day and night stage, neglecting dawn and dusk. Neglecting dawn and dusk is a critical oversight as the migration acts  a driver of ocean population dynamics, with a majority of predator-prey interactions taking place at dusk and dawn in the mixed layer, \citep{benoit2014critical}.

Advances in computational power and new modelling approaches have led to an exploration of models with continuous space and discontinuous time, \citep{jerome}, and models with continuous space and time \citep{verticalmigration}.
Expanding the complexity of the models allows prediction of fine-grained dynamics of the vertical migration, such as the exact location and size of the deep scattering layer, and the magnitude of feeding at dawn and dusk.

Models of poulation dynamics generally do not incorporate optimal behavior, with notable exceptions, \citep{Krivan1998,genkai2007macrophyte}. In particular, in previous models with a continuous spatial dimension, modeling population dynamics has been infeasible, \citep{pinti2019trophic}.


In this paper we present a new modelling approach for population games in continuous space, applied to the diel vertical migration. Our model is a modification of the model studied in \citep{verticalmigration}. We rephrase the vertical game as a linear complementarity problem, \citep{miller1991copositive}, which can solved efficiently. Our approach provides a unified framework for examining the population and behavorial time-scales.  Unifying the two time-scales allows us to examine how the vertical distribution of predators and prey change throughout the seasons and how this influences the population dynamics. We investigate the length and magnitude of the feeding rates of predators and consumers at dusk and dawn in spring, summer and autumn. In this way we model a the population dynamics associated to a single spawning cycle, and do not need to take ontogenics into account.

Organisms in game-theoretical models are usually seen as perfectly rational, acting on perfect state information. This seems patently unreasonable, as fish and zooplankton do not have perfect information on the state of the water column. In addition the minor gain in fitness from the almost-perfect choice to the perfect choice seems like it would be outweighed by the higher cognitive or sensorial cost of finding the perfect strategy.
 We incorporate this feature in our model by letting the animals maximize an expectation value with respect to their strategy, and letting their strategy incorporates noise. This allows us to examine how the optimal behavior with noise differs from that without noise, and how it changes the population dynamics. Again, this is feasible due to the numerical scheme we have chosen to examine the system. A change away from full rationality is expected to impact the fitness negatively, but it is unclear by how much. We examine this by looking at the population dynamics for the fully rational organisms compared to those with bounded rationality. As a baseline, we compare against the system with no behavorial optimization to see how the population dynamics evolve.




%Look at feeding dynamics, and compare the slow population dynamics with fast optimization, look at how optimal behavior changes the daily fluxes.


%Understanding advanced interest in undersntading the emergent patterns from the vertical game between predator and prey.


 %Iwasawa et al.

%Reference Jerome, UHTH
%is generic, and can easily be adapted to other situations.


%%
%ur method of solving the propose a novel method to solve continuous multi-population games, applying it to the concrete example of a predator-prey game.
