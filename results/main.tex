\documentclass[review,authoryear]{elsarticle}
\usepackage{changes}
\usepackage{pgf}
\usepackage{comment}
\usepackage[utf8]{inputenc}
\usepackage{graphics,epsfig}
\usepackage{subcaption}
\usepackage[T1]{fontenc}
\usepackage{float}
\usepackage{url}
\usepackage{srcltx}
\usepackage{hyperref}
\usepackage{amsmath,amsfonts ,amssymb,fancyhdr,setspace,lastpage,comment,graphics, cleveref,natbib, lineno}
\usepackage{mathrsfs}
\usepackage{todonotes}
\usepackage[all]{xy}




\DeclareMathOperator{\argmax}{argmax}
\newcommand{\pa}[1]{\left ( #1 \right )}
\newcommand{\abs}[1]{\left | #1 \right |}
\newcommand{\for}[0]{\text{ for }}
\newcommand{\overbar}[1]{\mkern 1.5mu\overline{\mkern-1.5mu#1\mkern-1.5mu}\mkern 1.5mu}
\newcommand{\ip}[2]{\left \langle #1,#2 \right \rangle}
\newcommand{\brak}[1]{\langle #1 \rangle}
\newcommand{\C}[0]{\mathbb{C}}
\newcommand{\Z}[0]{\mathbb{Z}}
\newcommand{\R}[0]{\mathbb{R}}
\newcommand{\norm}[1]{\left \Vert #1 \right \Vert }

\renewcommand{\part}[0]{\partial}

\let\oldphi\phi \let\phi\varphi \let\varphi\oldphi
\let\oldphi\phi \let\epsilon\varepsilon


%\title{Fast adaptive behavior in a tri-trophic system: Breakdown of the paradox of enrichment}
%\title{Solving multispecies population games in continuous space and time}

%\title{Solving multispecies population games in continuous space and time \tnoteref{t1}}\tnotetext[t1]{This document is the results of the researchproject funded by the National Science Foundation.}%\tnotetext[t2]{The second title footnote which is a longertext matter to fill through the whole text width andoverflow into another line in the footnotes area of thefirst page.}


%\author{Emil Friis Frølich, Uffe Høgsbro Thygesen}

\title{Solving multispecies population games in continuous space and time \tnoteref{t1}}
\tnotetext[t1]{This work was supported by the Centre for Ocean Life,
a Villum Kann Rasmussen Centre of Excellence supported
by the Villum Foundation.}
\author[1]{Emil F. Fr{\o}lich\corref{cor1}} %\fnref{fn2}
\ead{jaem@dtu.dk}

\author[2]{Uffe H. Thygesen}
\ead{uhth@dtu.dk}

\cortext[cor1]{Corresponding author}
%\fntext[fn1]{This is the first author footnote.}


\affiliation[1]{organization={Technical University of Denmark, Department of Applied Mathematics and Computer Science - DTU Compute},
     addressline={Building 303B, Matematiktorvet},
     postcode={2800},
     city={Kgs. Lyngby},
     country={Denmark}
     }
\affiliation[2]{organization={Technical University of Denmark, Department of Applied Mathematics and Computer Science - DTU Compute},
      addressline={Building 303B, Matematiktorvet},
      postcode={2800},
      city={Kgs. Lyngby},
      country={Denmark}
      }

\begin{document}



\begin{abstract}
  Game theory has emerged as an important tool to understand interacting populations in the last 50 years. Game theory has been applied to study population dynamics with optimal behavior in simple ecosystem models, but existing methods are generally not applicable to complex systems. In order to use game-theory for population dynamics in heterogeneous habitats, habitats are usually split into patches and game-theoretic methods are used to find optimal patch distributions at every instant. However, populations in the real world interact in continuous space, and the assumption of decisions based on perfect information is a large simplification. Here, we develop a method to study population dynamics for interacting populations, distributed optimally in continuous space. A continuous setting allows us to model bounded rationality, and its impact on population dynamics. This is made possible by our numerical advances in solving multiplayer games in continuous space. Our approach hinges on reformulating the instantaneous game, applying an advanced discretization method and modern optimization software to solve it. We apply the method to an idealized case involving the population dynamics and vertical distribution of forage fish preying on copepods. Incorporating continuous space and time, we can model the seasonal variation in the migration, separating the effects of light and population numbers. We arrive at  qualitative agreement with empirical findings. Including bounded rationality gives rise to spatial distributions corresponding to reality, while the population dynamics for bounded rationality and complete rationality are equivalent. Our approach is general, and can easily be used for complex ecosystems.
%\todo[inline]{Et aspekt mangler: Det med den beregningsmæssige teknik til at opløse det diskretiserede spil. Så tilføj en sætning eller to om det. Til gengæld kan nogle af sætningerne kortes lidt ned, hvis man går dem igennem med en tættekam: Abstracts bør være så koncise som muligt. Korrekt. }
\end{abstract}

\begin{highlights}\item An easily applicable general model for continuous-time population games in continuous habitats is introduced \item Population dynamics for bounded rationality are indistinguishable from population dynamics with perfect rationality \item Incorporating optimal behavior in a model of population dynamics partially explains seasonal variations in the diel vertical migration \end{highlights}

\begin{keyword} population dynamics \sep habitat choice \sep game theory \sep vertical migration \sep predator-prey %% keywords here, in the form: keyword \sep keyword
\end{keyword}


\maketitle


%\todo[inline]{Abstracted skal nok skrives om til sidst.}


\linenumbers{}
\modulolinenumbers[3]

\section{Introduction}
Population dynamics emerges from behavior of the animals; yet many models of population dynamics and ecosystems ignore behavior.  In the past 50 years, game theory has evolved into an invaluable tool for for including animal behavior in ecological models. Game theory gives a theoretical toolkit for understanding observed behavior and making predictions for how behavior will change in response to external changes. Game theory has been used to model a wide variety of situations where an animal needs to make a choice, from habitat choice \citep{krivan1997dynamic, kondoh2003foraging,kvrivan2008ideal}, mating behavior  \citep{rapoport1967exploiter}, and confrontation strategies \citep{smith1973logic}. The game theoretical models have proven successful, with empirical evidence backing up their validity as a model of animal behavior \citep{cooper1989communication,empirical_trait,behavioral_effects}.


%Game theory in biology emerged to model individual behavior, but individual behavioral changes compounds into large effects at the population scale. To incorporate behavior in population models
Behavior is fundamentally an individual phenomenon, and it is not obvious how to incorporate behavior in models of population dynamics with large numbers of interacting individuals. A reasonable assumption for interacting populations in an inhomogenous habitat, is that all animals seek to find the best spot simultaneously. For a single population this leads to the ideal free distribution \citep{fretwell1969territorial}. The population dynamics of a model where every individual is always at the best location is a population game \citep{kvrivan2009evolutionary}. The instantaneous population growth rates in a population game are determined by the instantaneous Nash equilibria of the individual habitat-choice game. Population games have emerged as a powerful tool to incorporate behavior in simple population models \citep{Krivan1998,genkai2007macrophyte, cressman2010ideal,pinti2021co, gonzalez2003dynamic}. However, the approach used in these models is not scalable to larger number of species or continuous habitats.


Population games are often simplified by only considering  one or two trophic levels, \citep{kvrivan2007lotka, sadowski2019predator}. This is in spite of e.g. mating behavior being influenced by the risk of predation, \citep{carranza1999red,lima2009predators}, naturally leading to a game with at least three types of players. Going to games with larger number of players can explain complex phenomena, which cannot be modeled with only two types \citep{pinti2019trophic}. Another simplification concerns the representation of space in habitat selection games.  Natural habitats often have continuous fitness gradients \citep{kawecki2004conceptual}, yet population games typically simplify this complex reality to a small finite number of patches, \cite{valdovinos2010consequences}. Population models in continuous space with multiple trophic levels or roles are generally hard to examine, as finding Nash equilibria in the resulting games is often prohibitively hard, \citep{empirical_trait,pinti2019trophic}. Resolving the issues of computing Nash equilibria quickly in a continuous setting allows the extension of population games to more realistic models. The question of whether to include behavior in a model or not becomes a question of relevance to the model rather than feasibility.


A critique of game-theoretical models is the assumption that players have perfect information and act in a perfectly rational manner, \citep{jones1999bounded}. Perfect information seems unreasonable, as animals do not have perfect state information \citep{simon1955behavioral}. In addition the minor gain in fitness from the almost-perfect choice to the perfect choice is often outweighed by the higher cognitive or sensorial cost of finding the perfect strategy \citep{simon1956rational, cohen2019bounded}. Though these concerns are well-founded, most models end up incorporating perfect rationality and information anyway. Classical satisficing models of bounded rationality cannot be verified empirically \citep{nonacs1993satisficing}, and with other attempts the \citep{bayesianmodel, thuijsman1995automata} the complexity has prevented the models from being implemented at the population level.


We introduce a method that allows the incorporation of behavior and imperfect decision making in population games in continuous space and time. The approach we introduce can readily be applied to study multi-species population dynamics emerging from a habitat-choice game in both continuous and discrete habitats. To illustrate the potential of the framework, we apply the method to diel vertical migration in the ocean. At dawn, billions of small fish and zoo-plankton migrate from the upper layers of the ocean to the deeper, darker layers, which can be directly measured as the deep scattering layer, \citep{sutton2013vertical, wang2014seasonal}. At dusk, the small fish and zoo-plankton migrate upwards. Here, we study the seasonal interplay between population dynamics and behavior. The model is an extension of the model studied in \citep{verticalmigration} to a population game. Our basic approach is to rephrase a continuous habitat selection game as a single linear complementarity problem, \citep{miller1991copositive}. We incorporate bounded rationality by requiring the strategies solve a diffusion equation, picking the strategy that maximizes the payoff with a given level of noise.
We couple the time scales of population dynamics and behavioral time scales, which allows us to examine how the vertical distribution of predators and prey change throughout the seasons and how this influences the population dynamics. We investigate the length and magnitude of the feeding rates of predators and consumers at throughout the day in spring, summer, and autumn of a single year. We examine how the optimal behavior with noise differs from that without noise, and how noise changes the population dynamics.

%%% Local Variables:
%%% mode: latex
%%% TeX-master: "main"
%%% End:

\section{Method}
\subsection*{General model framework}

The basic structure in the games we study is a system of $N$ players, with strategies given by probability distributions $\phi_i$ on a space $X$, henceforth an interval $[0,z_0]$. The payoffs $U_i$ of the players are given by matrices, or, in the continuous setting, linear operators $U_ij$ as

\begin{equation}
  \label{eq:utility}
  U_i = \sum_{j=1,j\neq i}^N \int n_{ij}\phi_i U_{ij} \phi_j dx%, ~i\neq j~n_{ij}=1,~i=j~n_{ij} = \frac{1}{2}
\end{equation}
We can see that this generalizes the notion of a polymatrix game, if the integral is replaced by a sum and the probability distributions are replaced by strategy vectors. The population dynamical systems that can be modelled via. games like this are where the population dynamics of species $i$ can be modelled as
\begin{equation}
  \dot{N_i} = N_i U_i
\end{equation}

In a polymatrix population game, there is polymorphic-monomorphic equivalence \citep{broom2013game}, so an individual of type $j$ cannot distinguish whether an opposing population playing a mixed strategy $\phi$ is a mixture of individuals playing pure strategies, or all individuals play a single mixed strategy. As such, our approach entails each individual in a species or population making optimal choices, and the other groups reacting to the total choice.

The Nash equilibrium of such a system consists of a family of probability distributions $(\phi_i^{*,NE})_{i=1}^N$ where no player can increase their utility by unilaterally deviating from their strategy.% An example of such a game is a habitat selection game, where picking the strategy that maximizes the instantaneous growth gives rise to an ESS \citep{kvrivan2009}.

\subsubsection*{Noisy strategies}
Our model incorporates that players are not necessarily perfectly rational: The player may not be a perfect decision-maker, due to imperfect information or limited capacity of information processing, but it can also model errors in our perception of the player's objectives, or inability to actuate a decision perfectly, for example due to turbulence in the water column. Our model of imperfect rationality is as follows: Say that the player aims to play the strategy $f_X(\cdot)$, which is a probability density function on $[0,z_0]$. Then our model posits that the player actually plays a strategy $\phi_i(\cdot ,\sigma)$, which is a smoothed version of $f_X(\cdot)$ obtained by solving the initial value problem
\begin{align}
  \label{eq:density_PDE}
  &\partial_s \phi_i = \frac{1}{2}\partial_z^2 \phi_i \\
  &\partial_z \phi_i \mid_{z=0} = 0 \\
  &\partial_z \phi_i \mid_{z = z_0} = 0 \\
  & \phi_i(z,0) = f_X(z) \quad .
\end{align}
on the interval $[0,\sigma]$. Thus, the parameter $\sigma$ determines the degree of smoothing: With $\sigma=0$, the player is perfectly rational ($\phi_i(z,0)=f_X(z)$) while with $\sigma=\infty$, we have a completely random decisions where $\phi_i(z,\infty)$ is a constant function of $z$, corresponding to a uniform distribution on $[0,z_0]$. Note that $s$ or $\sigma$ are not connected to time; this smoothing takes place instantaneously at each point in time.

Numerically, this smoothing is performed by first determining the fundamental solution to this initial value problem, ignoring boundaries, which is a Gaussian kernel. Then the boundary conditions are implemented using the method of images. Finally, the initial condition is convolved with this kernel.


\subsubsection*{Spatial discretization}
In order to calculate the Nash Equilibrium efficiently, and perform numerical integration precisely we discretize the interval $[0,z_0]$ with a spectral scheme based on Legendre polynomials, \citep{kopriva2009implementing}. This allows precise integration and differentation with only relatively few points.
We approximate pure strategy of being in a point $z_i$  by a normalized hat-function $e_i$, zero everywhere apart from $z_i$.
\begin{align*}
	& \int_{z_i}^{z_{i+1}} e_i dz = 1 \\
	&e_i(z_{i-1}) = 0,~ e_i(z_{i+1}) = 0
\end{align*}
Working on a grid with $M$ points, a strategy then becomes a linear combination of hat-functions,
\begin{align*}
  &\phi_{i} = \sum_{j<M} a_{j,i} e_j, \quad i\in \{1,\dots, N\} \\
  &\sum_{j<M} a_{j,i} = 1 \quad i\in \{1,\dots, N\}
\end{align*}
The strategy of a player is fully determined by the $a_i$'s.

When considering non-optimal actors, we need to implement the convolution with $f_Y$, which also assures that the resulting distrbution is smooth. An added benefit of incorporating bounded rationality then becomes that our strategy profiles are guaranteed to be smooth, decreasing the number of points needed for exact evaluation of the integrals.


\subsubsection*{Finding the Nash Equilibrium}
Finding the Nash Equilibrium in a game in continuous space is usually a hard task, requiring the development of bespoke methods, \citep{verticalmigration}, or very long runtimes, \citep{jerome}. The method we have use circumvents these problems, by combining a little-known result in mathematical optimization with a spectral scheme.

By discretizing space, we have reduced an uncountable strategy set to a more manageable finite amount, with pure strategies $e_k$. The gain of a player playing strategy $e_k$ against player $j$ playing strategy $e_l$ can be determined as $A_{ij}(e_k,e_l)$, \Cref{eq:utility}. The discretization allows us to write up payoffs for a finite approximation version of the continuous game,  with entry $(k,l)$ determined through $\ip{e_k}{A_{ij}e_l}, k,l \in \{1,\dots M\}$.
Our discretization has reduced the problem to a bimatrix game, where finding the Nash equilibrium is more tractable.

It does not appear to have diffused through the literature, but a Nash equilibrium of a polymatrix game can be found by solving a single linear complementarity problem \citep{miller1991copositive}. Using a modification of the argument from \citep{miller1991copositive}, specialized to the case of two-player (bimatrix) games but easily generalizable to the general $n$-player case. Assume that $(s^*_1,s^*_2)$ constitute a Nash equilibrium in mixed strategies with values $\gamma_1 = \ip{s^*_1}{E_1 s^*_2}$ and  $\gamma_2 = \ip{s^*_2}{E_2 s^*_1}$ to the consumer and predator, respectively. Then
\[
  \ip{s_1}{1_n} =
  \ip{s_2}{1_n} =
  1
\]
since these mixed strategies are probability distributions on strategy space. Here $1_n$ is a vector of ones. In addition the Nash equilibrium dictates
\[
  E_1 s_2 = 1_n \gamma_1 - w_1
  ,\quad
  E_2 s_1 = 1_n \gamma_2  - w_2
\]
$w_1$ and $w_2$ are non-negative ``slack variables'' that state that the payoff for the first player can be no greater than the expected payoff $\gamma_1$, but can be smaller for some fixed strategies. These non-optimal strategies, where the slack $w_1$ is positive, must then be chosen with probability 0, and as a consequence the complementarity condition
\[
  \ip{s^*_1}{w_1} =   \ip{s^*_2}{w_2} = 0
\]
holds. Assume for convenience that all elements in $E_1$ and $E_1$ are negative; this can always be obtained without changing the Nash equilibria by substracting a constant from $E_1$ and $E_2$. Consequenty, also the payoffs $\gamma_1$ and $\gamma_2$ are negative and thus the vector $z = (s_1,s_2,-\gamma_1,-\gamma_2)$ satisfies the Linear Complementarity Problem (LCP)
\[
\label{eq:lcp}
  z \geq 0,
  w \geq 0 ,
  H
  z
  +
  \left(
    \begin{array}{c}
      0 \\
      0 \\
      -1 \\
      -1
    \end{array}
  \right)
  =
  w
  ,
  \quad
  \ip{z}{w} = 0
  .
\]
where
\[
  H =
  \left[
    \begin{array}{cccc}
      0 & -E_1 & -1_n & 0 \\ -E_2 & 0 & 0 & -1_n \\
      1_n & 0 & 0 & 0 \\
      0 & 1_n & 0 & 0
    \end{array}
  \right]
\]
Conversely, assume that $z=(s_1,s_2,\gamma_1,\gamma_2)$ and $w$ solve the Linear Complementarity Problem, then it is straightforward to see that the mixed strategies $(s_1,s_2)$ form a Nash equilibrium with values $(\gamma_1,\gamma_2)$. The assumption that $E_1$ and $E_2$ have negative elements imply that the matrix $H$ is copositive plus (meaning, for all $z\geq0$ with $z\neq0$ it holds that $\ip z{Hz}>0$) which assures that the LCP to has a solution, in particular through Lemke's algorithm.

Solving \Cref{eq:lcp} was done through two different methods. The interior-point method as implemented in IPOPT, \citep{wachter2006implementation}, called via. the auto-differentation software CasADi \citep{Andersson2019}, and Lemkes Algorithm implemented in the Numerics package in Siconos, \citep{acary2019introduction}. Experience showed that Lemkes algorithm was the fastest.

%%% Local Variables:
%%% mode: latex
%%% TeX-master: "main"
%%% End:


\section{Modeling population dynamics and the diel vertical migration}
We apply our method to the diel vertical migration of oceanic animals where the game is well-understood, but the interplay between the daily variations and the population dynamics have not been properly investigated. %In addition the Nash Equilibrium is known to be unique \citep{verticalmigration} at every instant.
We consider a food-chain in a water column, consisting of a resource $R$, a consumer $C$, and a predator $P$. The resource is thought of as phytoplankton, the consumer as copepods and the predator as forage fish. The predators and consumers are each distributed in the water column according to probability distributions, $\phi_c(z,t),\phi_p(z,t)$, and the resource is distributed according to $r(z,t)$.

Forage fish are visual predators, so their predation success is heavily light dependent. The available light decreases with depth in the water column, and varies with the time of day.
The light intensity $I$ at depth $z$ is approximately $I(z) = I_0\exp(-kz)$, and the light-dependent clearance rate of a predator is $\beta_{p,0}$.  However, even when there is no light available there is still a chance of catching a consumer if it is directly encountered,  so the clearance rate, $\beta_p(z,t)$, of forage fish never goes to 0 even at the middle of the night or at the deepest depths.
\begin{equation*}
  \beta_p(z,t) = \beta_{p,0} \frac{I(z,t)}{1+I(z,t)} + \beta_{p,min}
\end{equation*}


We model the light-levels at the surface via. the python package pvlib \citep{holmgren2018pvlib} in the North Sea. The light levels are given by the direct horizontal light intensity at the sea-surface, neglecting more complicated optic effects. The model takes the precipitable water $w_a$, and aerosol optical depth, $aod$. We model light decay throughout the water column as $\exp(-kz)$.


In contrast to forage fish, copepods are olfactory predators, and their clearance rate, $\beta_c$, is essentially independent of depth and light levels.
\begin{align*}
	\beta_c(z,t) &=  \beta_{c,0}
\end{align*}

The interactions between the consumer and resource are local, as are the interactions between a predator and a consumer. The local encounter rate between consumers and resources is given by $C\beta_c(z,t)\phi_c(z,t)r(z,t)$, and the local encounter rate between predators and consumers is $CP\beta_p(z,t)\phi_c(z,t)\phi_p(z,t)$.

\subsection{Population dynamics}

The resource cannot move actively, so its time dynamics are naturally specified locally. The growth of the resource is modeled with a logistic growth, with a loss from grazing by consumers and diffusion from the natural movement of the water. We assume interactions can be described with a Type I functional response, allowing us to eventually use the method developed in \Cref{sec:gen_model}. The resource dynamics become:
\begin{equation}
  \label{eq:res_dyn}
	\dot{r} = r(z,t)\pa{1-\frac{r(z,t)}{r_{max}(z)}} - \beta_c(z,t)\phi_c(z,t)C(t) r(z,t)  + k \partial_z^2 r(z,t) \\
\end{equation}
%In natural environments, undersaturation of nutrients is the norm, \citep{}.

The total population growth of the consumer population is found by integrating the local grazing rate over the entire water column multiplied by a conversion efficiency $\epsilon$, subtracting the loss from predation. The growth of the predators is given by the predation rate integrated over the water column. The instantaneous pr. capita population growth without metabolic losses, of the consumer $F_c$ and predator $F_p$  become:
\begin{equation}
  \begin{split}
	F_c(\phi_c, \phi_p) &= \int_0^{z_0} \varepsilon \beta_c(z,t)\phi_c(z,t)r(z,t) dz\\ &- P(t)\int_0^{z_0} \beta_p(z,t) \phi_c(z,t) \phi_p(z,t)dz \\
	F_p(\phi_c, \phi_p) &=  C(t) \int_0^{z_0} \varepsilon \beta_p(z,t)\phi_c(z,t)\phi_p(z,t) dz
  \end{split}
  \label{eq:fitness}
\end{equation}


Using \Cref{eq:fitness} we arrive at equations for the predator-prey population dynamics:
\begin{equation}
  \begin{split}
	\dot{C} &= C(t)\left ( F_c - \mu_C \right ) \\
	\dot{P} &= P(t) \left ( F_p - \mu_P  \right )
\end{split}
  \label{eq:population_growth_prob_dens}
\end{equation}
We use $F_c,~F_p$, \Cref{eq:fitness}, as our fitness proxies.
\subsection{Simulating the model}

%The instantaneous fitness pr. capita of a forage fish $(F_p)$ or copepod $(F_c)$ is given by t. We arrive at the fitness by dividing the population growth rate \Cref{eq:population_growth_prob_dens} by the total populations, eliminating the terms $C(t), P(t)$ outside the parentheses in \Cref{eq:population_growth_prob_dens}.

%\todo[inline]{Fitness er ikke det samme som specifikke vækstrater. }

%\todo[inline]{Når nu disse størrelser er indført, hvorfor så ikke bruge dem over det hele? Så ville mange formler være meget mere kompakte.}

As in \Cref{sec:gen_model} at any instant, all consumers and predators simultaneously seek to find the strategy that maximizes their fitness ($F_c, F_p$) \Cref{eq:utility}. A strategy in our case is a square-integrable probability distribution in the water column, i.e. an element in $K$, \Cref{eq:space_of_dists}.

%The optimal strategy $\phi_c^*$ of a consumer depends on the strategy of the predators, and likewise for $\phi_p^*$ for the predators. the space of square-integrable probability distributions on $[0,z_0]$ by $K$ (\Cref{eq:space_of_dists}), this can be expressed as:
%\begin{align*}
%	\phi_c^*(z,t)(\phi_p) &= \argmax_{\phi_c \in K}  F_c(\phi_c, \phi_p)  \\
%	\phi_p^*(z,t)(\phi_c) &= \argmax_{\phi_p \in K} F_p(\phi_c, \phi_p)
%\end{align*}
Using the notation of \Cref{sec:gen_model}, the Nash equilibrium of the instantaneous game is:
\begin{equation}
  \label{eq:nash_equilibria}
  \begin{split}
  	\phi_c^{*,NE} &=  \argmax_{\phi_c \in K}  F_c(\phi_c, \phi_p^{*,NE}) \\
  	\phi_p^{*,NE} &=  \argmax_{\phi_p \in K} F_p(\phi_c^{*,NE}, \phi_p)
  \end{split}
\end{equation}
We apply the method \Cref{eq:lcp} to find the Nash equilibrium of the discretized system. Using the Nash equilibrium \Cref{eq:nash_equilibria} we are able to solve the time-dynamics for the predator-prey system \Cref{eq:population_growth_prob_dens} by a Euler scheme. The dynamics of the resource are more complicated due to the diffusion term, \Cref{eq:res_dyn}. We solve the partial differential equation for the resource using the method of exponential time-differencing \citep{hochbruck2010exponential} with a first-order approximation of the integral. Using exponential time-differencing guarantees a stable solution, though the system may be stiff \cite{hochbruck2010exponential}.

\subsection{Model parameters}
Following \citep{yodzis1992body}, we parameterize the clearance and loss rates in a metabolically scaled manner following Kleiber's law, \citep{yodzis1992body}, using scaling constants from \citep{kha_2019}. We use the default parameters in the clear-sky model, modeling a sequence of moonless nights. This is a bit of a simplification, but it should not have a great effect on our results. The North Sea is modeled with a rather high attenuation coefficient. We use the notation $\mathcal{N}(0,\sigma^2)$ for the normal distribution with mean $0$ and variance $\sigma^2$.


\begin{tabular}{l  l  l}
  Precipitable water & $w_a$ & 1 g $\cdot$ m$^{-3}$\\
  Aeorosol optical depth & $aod$ & 0.1 \\
  Light decay & $k$ & 0.1 m$^{-1}$\\
  Ocean depth & $z_0$ & 90 m \\
  Consumer mass & $m_c$ & 0.05 g \\
  Predator mass & $m_p$ & 20 g \\
  Consumer clearance rate & $\beta_c$ & 32 m$^{3}$ year$^{-1}$ \\
  Predator clearance rate & $\beta_{p,0}$ & 2800 m$^3$ year$^{-1}$ \\
  Consumer metabolic rate & $\mu_c$ & 0.24 year$^{-1}$ \\
  Predator metabolic rate & $\mu_p$ & 21 year$^{-1}$ \\
  Minimal attack rate & $\beta_{p,min}$ & $5 \cdot 10^{-3} \beta_p$ \\
  Phytoplankton growth & $\lambda$ & 100 year$^{-1}$ \\
  Phytoplankton max & $r_{max}$ & $10\mathcal{N}(0,6)$ g m $^{-3}$ \\
  Irrationality & $\sigma$ & 10 m$^2$ \\
  Diffusion rate & k & 500 m$^{2}$ year$^{-1}$ \\
  Initial consumers & $C_0$ & 4  g m$^{-2}$ \\
  Initial predators & $C_0$ & 0.04  g m$^{-2}$ \\
  Initial resources & $r_0$ & 4$\mathcal{N}(0,6)$ g m$^{-3}$
\end{tabular}

%%% Local Variables:
%%% mode: latex
%%% TeX-master: "main"
%%% End:

\section{Results}

%\begin{figure}
%\includegraphics{heatmapsday10_nonrandom.pdf}
%\caption{Vertical distribution of consumers \emph{(1)} and predators \emph{(2)} throughout the 1st of May, in hours from noon.}
%\end{figure}

\begin{figure}
\includegraphics{heatmapsday90_nonrandom.pdf}
\caption{Vertical distribution of consumers \emph{(1)} and predators \emph{(2)} throughout the 1st of July. The time is in hours from noon. }
\label{fig:heatmaps_90_nonrandom}
\end{figure}
The vertical migration of consumers, \Cref{fig:heatmaps_90_nonrandom}(1) is clear here in the middle of the summer.They are highly concentrated at the top of the water column during nighttime, and at day they scatter throughout the deep. The pattern of the predators is slightly different from the consumer pattern, \Cref{fig:heatmaps_90_nonrandom}. At nighttime there is still a non-zero concentration of predators in the upper layers of the water-column, there to catch any errant prey.
\begin{figure}
\includegraphics{heatmapsday180_nonrandom.pdf}
\caption{Vertical distribution of consumers \emph{(1)} and predators \emph{(2)} throughout the 1st of October. The time is in hours from noon.}
\label{fig:heatmaps_180_nonrandom}
\end{figure}
Moving the hands on the clock forward to October, we again see a clearly defined vertical migration, \Cref{fig:heatmaps_180_nonrandom}. The migration differs from the previous migration, in that the descent and ascent are steeper, and the distributions are wider during the night.

\begin{figure}
\includegraphics{populations.pdf}
\caption{Total populations of consumers \emph{(blue)}, predators, \emph{(red)} and resources \emph{(purple)} from 1st of april to 1st of october. We vary the rationality, from total rationality \emph{(1)}, bounded rationality ($\sigma = 10...$), \emph{(2)} and fully irrational, $\sigma = \infty$, \emph{(3)}.}
\label{fig:long_term_populations}
\end{figure}
The the difference in population dynamics between a system with no behavioral optimization, \Cref{fig:long_term_populations}(3), bounded rationality \Cref{fig:long_term_populations}(2) and full rationality \Cref{fig:long_term_populations}(1) is stark. The resources reach a stable level quickly in all three cases, but the populations of consumers and predators differ markedly. The difference in populations between the system with bounded rationality \Cref{fig:long_term_populations}(2) and the fully rational system appears to be negligible, \Cref{fig:long_term_populations}(1). The main driver seems to be the ability to retreat to a refuge, and not exactly how it happens.
%Large differences between irrational and the rational possibilities
\begin{figure}
\includegraphics{growth_short_rational.pdf}
\caption{Seasonal comparison of consumer \emph{(blue)} and predator, \emph{(red)} feeding patterns on 1st of May \emph{(1)}, 1st of July \emph{(2)} and 1st of October \emph{(3)}}
\end{figure}

\begin{figure}
\includegraphics{pop_short_rational.pdf}
\caption{Seasonal comparison of consumer \emph{(blue)} and predator, \emph{(red)} pr. capita growth patterns on 1st of May \emph{(1)}, 1st of July \emph{(2)} and 1st of October \emph{(3)}}
\label{fig:pop_short_term}

\end{figure}
Clear emergence
\begin{figure}
\includegraphics{specific_dists_rational.pdf}
\caption{Daily distribution of consumers \emph{blue} and predators \emph{red} at midnight \emph{(1)}, noon \emph{(2)} and at 18:45, \emph{3} with full rationality}
\label{fig:specific_dists_rational}
\end{figure}

Tracking, preventive
\begin{figure}
\includegraphics{specific_dists_semirational.pdf}
\caption{Daily distribution of consumers \emph{blue} and predators \emph{red} at midnight \emph{(1)}, noon \emph{(2)} and at 18:45, \emph{(3)} with bounded rationality}
\label{fig:specific_dists_irrational}
\end{figure}

Smoother, more realistic

\section{Discussion}
\subsection*{What we found} %1 Paragraph
%The game we study is a mean field game (MFG), where the self-interaction is mediated by the other player. Our game has polymorphic-monomorphic equivalency due to the bilinear nature. Mono-Poly is a necessary requirement for models of DVM, since not all copepods/fish behave exactly the same, and we thereby encapsulate minor variations in each fish.

%Short term static driving long term variations
%Strength of numerical approach
%Seasonal variation, driven by population dynamics
%Bounded rationality is the same at long-term, quite different short-term
Our study of a Lotka-Volterra system with optimal behavior in the water column focused on both the emergent behavior and population dynamics. We saw the emergence of populations varying heavily over the long term, from a system where the populations appear static when inspected at short time-scales. The short-term dynamics show a clear distinction between predator and consumer feeding modes, with most predator-prey interactions happening at dusk and dawn. By comparing the results for bounded rationality and unbounded rationality on the long and short time scales, we see two different pictures emerge. The behavior predicted by the two different models are quite different, with a distinct difference in the optimal distributions in the water. Though the behavior is different, the difference averages out over the long time-scale, and the population dynamics for populations with bounded and unbounded rationality are indistinguishable. The results illustrate the strength of our numerical approach to studying population games, and underline the importance of robust algorithms and discretization schemes.


\subsection*{Interpretation of the findings and what they mean} %1-2 paragraphs
%Clear emergence of eeding times, shows importance fo dusk and dawn, maybe possible to average out
%Incorporating populations and threats necessary to specify migration patterns
%The choice of numerical approach is of critical importance, upgrading numerics can lead to new insights
%Unbounded rationality, though unreasonable empirically, is not so important on the large scale but very important on small scale.
Our model illustrates that there is a complicated interplay between population dynamics and seasons, and though the light levels may be the same at two different times of the year, the migration patterns can differ radically due to the differences in populations. Models incorporating the vertical migration as an essential component need to consider the population levels as well as the light levels.  



\subsection{Compare results to litterature} %3-5 paragraphs
That vertical migrations change seasonally due to changes in nutrient and light has been studied in the arabian sea, \citep{wang2014seasonal}, and our method enables testable predictions of how this seasonal migration can vary, allowing comparison with large empirical studies \citep{klevjer2016large}.
The shapes of the distributions at both day and night seem to agree better with observations in the model with bounded rationality, \citep{hay1991zooplankton}, lending credence to the idea that copepods are not perfectly rational or subject to turbulence \citep{visser2001observations}.
In short-term models, constant populations are typically assumed constant \citep{...,...,...}. Though the populations change dramatically over the long term in our model, the population levels are essentially unchanged on a daily basis. The proposed population-dynamical model passes an essential test, as wildly fluctuating populations in the short term cannot represent the underlying physical reality. In contrast, the near-constant short-term population emerges from the behavorial optimization in the model when comparing to the model with no behavior.

When looking at ecosystems, there are two main approaches to incorporating behavior:
\begin{enumerate}
  \item Assume perfect rationality
  \item Ignore it
\end{enumerate}
The middle way, where animals take rational choices but with a bound on their rationality is typically not included. Imperfect rationality captures that animals might not be perfectly aware of the state of a system, or when distinguishing between almost-equivalent options, choosing the best might require a disprortionate effort. The distribution of copepods in the water column with bounded rationality closely resembles empirical distributions, \citep{thomasetal}, as the model with perfect rationality suffers from an unphysical discontinuity, also seen in \citep{uhth}. Comparing the distributions with perfectly rational behavior to the ones with bounded behavior, we see that the day-time distribution is essentially the same. The optimal behavior at day is to be spread out, and the loss in fitness from the more spread-out night-ime distribution does not appear to be significant, when comparing the long-term population trajectories.

\subsection{Limitations and generalizability} %1-2 paragraphs
The cost of migration is not incorporated in the model, which it could be. Howver, the cost of mirgation could also be introduced through a proxy where there is a constant species-specific penalty-field incentivizing specific depths, eg. disallowing infinetly deep migrations.
The model we have developed is ready-made for incorporating more species. A logical next step would be to examine the concept of cascading migrations and Vinogradovs ladder. Our approach essentially only depends on the linearity of the fitness proxy, so it could be imagined that ontogenics could be incorporated, via. structuring the populations and considering fitness based on the Leslie matrix, enabling a proxy for life-cycle optimization.

As short feeding bouts are the drivers of the population dynamics, it becomes quite hard to formulate a long-term model of the slow population dynamics without taking into account the fast feeding dynamics, as these are strongly non-linear and feedback-driven


\subsection{Further work} %1 paragraph
The driver of seasonal variations in migration patterns, is a strong point of the model. Clear variatinos in the expected distributions stand out, and these can theoretically be tested via. echo-acoustics.
The boundedly rational distributions we arrive at with our model for zooplankton and forage fish seem to agree qualitively with observations to a relatively degree. As such, our model provides a tentative answer to how much realism is lost by the usual assumption of complete rationality. The results we portray show that the overall patterns agree between a population with bounded rationality and one which is completely rational. The usual assumption of complete rational seems to be reasonable modelling assumption, but care should be taken. We show that it is possible to include bounded rationality in a systematic fashion, allowing it as a tuning parameter in future models.
Seasonal variation in plankton production.


The fitness proxy we use guarantees an ESS, \citep{krivan}. Combing our modeling approach with these theoretical results, it becomes feasible to model multi-actor multi-environment ecosystems. A weakness of the model is, of course, that it does not incorporate a concept of satiation nor ontogenics. \todo[inline]{Til huskelisten: Det med entydighed af Nash-ligevægte. Om ikke andet må vi nævne det som et emne, der kræver opmærksomhed - hvis vi anbefaler numerisk analyse til at finde ligevægtene, så er der også brug for numeriske løsninger til at afgøre entydighed.}.

\subsection{Conclusion} %1 paragraph

When looking at the population dynamics of a predator-prey system through half a, the seasonal impacts stand out. Even though our model is simple in nature, it can catch essential features as the spring-bloom, and varying phytoplankton throughout the seaonss.  The long-term patterns are invisible when investigating the short-term fluctuations. As we can see the primary trophic interactions happens in a very short time, the system is fundamentally a slow system driven by a fast underlying dynamic. The migration patterns that we find with our model agree with  recent models of vertical migrations, also based on a game-theoretical perspective. This strengthens the conclusions of all three model families, as they are based on fundamentally different numerical and algorithmic schemes.

%\input{conclusion.tex}
\subsection*{Declarations}
\subsubsection*{Funding}
This work was supported by the Centre for Ocean Life,
a Villum Kann Rasmussen Centre of Excellence supported
by the Villum Foundation.
\subsubsection*{Code availability}
All code for reproducing the results of this project is available on github \url{https://github.com/jemff/food_web}.
\subsubsection*{Conflict of interest}
The authors declare that they have no conflict of interest.
\subsubsection*{Authors' contributions}
E.F.F. and U.H.T. designed the study. E.F.F. realized the model design. E.F.F. coded the model and chose the numerical approaches. E.F. analyzed the results with the assistance of U.H.T.  E.F.F. wrote the paper with contributions from U.H.T. authors. All authors read and approved the final version.
\subsubsection*{Data availability}
All data can be generated using the file data_lemke_generator.py  from the git repository \url{https://github.com/jemff/food_web}.


\bibliographystyle{elsarticle-harv}
\bibliography{bibliography}

\end{document}
