
\begin{abstract}
  Population dynamics are generally modelled without taking behavior into account. This in spite of the largest daily feeding times for predators, namely at dawn and dusk, being driven by behavior. The daily pattern stems from the Diel Vertical Migration (DVM) in the aquatic setting and crepuscular behavior in the terrestrial setting. This is usually explained by prey avoiding visual predators, and visual predators seeking to find prey. We develop a game-theoretical model of predator-prey interactions in continuous time and space, finding the Nash equilibrium at every instant. By unifying results for the general resolution of polymatrix games, and a spectral discretization scheme, we can resolve the spatially continuous game nearly instantaneously. Our approach allows a unified model for the slow time-scale of population dynamics, and the fast time-scale of the vertical migration, under seasonal changes. We use the diel vertical migration as a case, examining emergent phenomena from the introduction of the fast dynamics.
  On the behaviorial time-scale, we see the emergence of a deep scattering layer from the game dynamics. On the longer time-scale of population dynamics, the introduction of optimal behavior has a strong stabilizing, compared to the model without optimal behavior. In a changing seasonal environment, we observe a change in daily migration patterns throughout the seasons, driven by changes in both population and light levels. The framework we propose can easily be adapted to population games in inhomogenous terrestrial environments, and more complex food-webs.
\end{abstract}

\section{Introduction}

\todo[inline]{Jeg synes at fokus skal drejes over på ``emergent population dynamics with adaptive behavior'' og at DVM ``bare'' er en illustrerende case''. E: Gjort i Abstract, nu mangler selve introduktionen}

Behavior is an inescapable part of lives of animals, both aquatic and terrestrial. In spite of this, models of population dynamics typically do not incorporate behavior. Instead, populations are modeled as interacting chemical through functional responses, or going through life-cycles described by a Leslie matrices. There has been some work on incorporating game theory in models population dynamics with habitat selection, \citep{krivan,idealfreedistribution}. Models of poulation dynamics generally do not incorporate optimal behavior, with notable exceptions, \citep{Krivan1998,genkai2007macrophyte}. In particular, in previous models with a continuous spatial dimension, modeling population dynamics has been infeasible, \citep{pinti2019trophic}. This is problematic, the behaviorially driven diel vertical migration acts a driver of ocean population dynamics, with a majority of predator-prey interactions taking place at dusk and dawn in the mixed layer, \citep{benoit2014critical}.
In spite of this, no general approach has been advanced, available and easy to use for modelling any system, be it a system of interacting species or a mating game. Instead, a natural course is often to take a heuristic approach to eg. habitat selection in an ecosystem game when considering population dynamics, \citep{ho2019}, rather than considering the emergent population dynamics from adaptive behavior.

%Add more about how behavior leads to changes in population dynamics, following paper 1. 

Game theory in general has made huge inroads in describing large parts of animal behavior,  from habitat-choice, \citep{idealfreedistribution}, mating behavior  \citep{battleofthesexes}, and confrontation startegies \citep{hawkdove}. These factors are usually entirely ignored in population dynamical models, especially with a plethora of interacting species, since an increase in species richness leads to polymatrix games rather than simple symmetric or bimatrix games.

In this paper we present a new modelling approach for incorporating game theory into population dynamics, illustrated with an application to the diel vertical migration. Our model is a modification of the model studied in \citep{verticalmigration}. Our basic approach is to rephrease a continuous habitat selection game as a linear complementarity problem, \citep{miller1991copositive}, which can solved efficiently. The approach we propose can readily be applied to any polymatrix game, as well as any multi-agent game in continuous space. In essence, our approach provides a unified framework for examining the population and behavorial time-scales.  Unifying the two time-scales allows us to examine how the vertical distribution of predators and prey change throughout the seasons and how this influences the population dynamics. We investigate the length and magnitude of the feeding rates of predators and consumers at dusk and dawn in spring, summer and autumn. In this way we model a the population dynamics associated to a single spawning cycle, and do not need to take ontogenics into account.

Organisms in game-theoretical models are usually seen as perfectly rational, acting on perfect state information. This seems unreasonable, as animals do not have perfect state information. In addition the minor gain in fitness from the almost-perfect choice to the perfect choice seems like it would be outweighed by the higher cognitive or sensorial cost of finding the perfect strategy.
 We incorporate this feature in our model by letting the animals maximize an expectation value with respect to their strategy, and letting their strategy incorporates noise. This allows us to examine how the optimal behavior with noise differs from that without noise, and how it changes the population dynamics. Again, this is feasible due to the numerical scheme we have chosen to examine the system. A change away from full rationality is expected to impact the fitness negatively, but it is unclear by how much. We examine this by looking at the population dynamics for the fully rational organisms compared to those with bounded rationality. As a baseline, we compare against the system with no behavorial optimization to see how the population dynamics evolve.




%Look at feeding dynamics, and compare the slow population dynamics with fast optimization, look at how optimal behavior changes the daily fluxes.


%Understanding advanced interest in undersntading the emergent patterns from the vertical game between predator and prey.


 %Iwasawa et al.

%Reference Jerome, UHTH
%is generic, and can easily be adapted to other situations.


%%
%ur method of solving the propose a novel method to solve continuous multi-population games, applying it to the concrete example of a predator-prey game.

%%% Local Variables:
%%% mode: latex
%%% TeX-master: "main"
%%% End:
