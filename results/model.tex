
\section{Model}

\subsubsection*{Model introduction}
We consider a food-chain in a water column, consisting of a resource $R$ with concentration $r(z,t)$, a consumer $C$ with concentration $c(z,t)$ and a predator with concentration $p(z,t)$. The resource is thought of as phytoplankton, the consumer as copepods and the predator as forage fish. The concentrations and total amounts are related as:
\begin{align}
        R(t) &= \int_0^{z_0} r(z,t) dz \\
	      C(t) &= \int_0^{z_0} c(z,t) dz \\
	      P(t) &= \int_0^{z_0} p(z,t) dz
\end{align}
Forage fish are visual predators, so their predation success is heavily light dependent. The available light decreases with depth in the water column, and varies with the time of day.
The light intensity $I$ at depth $z$ is approximately $I(z) = I_0\exp(-kz)$, and the basic clearance rate of a predator at maximum light is $\beta_{p,0}$.  However, even when there is no light available there is still a chance of catching a consumer if it is directly encountered,  so the clearance rate, $\beta_p(z,t)$, of forage fish never goes to 0 even at the middle of the night or at the deepest depths.
\begin{align}
  \beta_p(z,t) = \beta_{p,0} \frac{I(z,t)}{1+I(z,t)} + \beta_{p,min}
\end{align}


We model the light-levels at the surface via. the python package pvlib, \citep{}, using a simple Clear Sky model in Oresund between Denmark and Sweden. The light levels are given by the direct horizontal light intensity at the sea-surface, neglecting more complicated optic effects. The model takes the precitibale water $w_a$, and aerosol optical depth, $aod$. We model light decay throughout the water column as $\exp(-kz)$.


In contrast to forage fish, copepods are olfactory predators, \citep{}, and their clearance rate, $\beta_c$, is essentially independent of depth and light levels, \citep{}.
\begin{align}
	\beta_c(z,t) &=  \beta_{c,0}
\end{align}

The interactions between the consumer and resource are local, as are the interactions between a predator and a consumer. The local encounter rate between consumers and resources is given by $\beta_c(z,t)c(z,t)r(z,t)$, and the local encounter rate between predators and consumers is $\beta_p(z,t)c(z,t)p(z,t)$.

\subsubsection*{Population dynamics}
The resource cannot move actively, so its time dynamics are naturally specified locally. The growth of the resource is modeled with a logistic growth, with a loss from grazing by consumers and diffusion from the natural movement of the water.
To simplify the model, we assume interactions can be described with a Type I functional response. In natural environments, undersaturation of nutrients is the norm, \citep{}.


The total population growth of the consumer population is found by integrating the local grazing rate over the entire water column multiplied by a conversion efficiency $\epsilon$, subtracting the loss from predation. The growth of the predators is given by the predation rate integrated over the water column:
%Lotka-Volterra:
\begin{align}
  \label{eq:population_growth}
  \dot{r} &= r(z,t)\pa{1-\frac{r(z,t)}{r_{max}(z)}} - \beta_c(z,t)c(z,t) r(z,t) + k \partial_x^2 r(z,t) \\
  \dot{C} &=  \int_0^{z_0} \varepsilon \beta_c(z,t)c(z,t)r(z,t) dz- \int_0^{z_0} \beta_p(z,t) c(z,t) p(z,t) dz - C(t)\mu_C  \\
  \dot{P} &=  \int_0^{z_0} \varepsilon \beta_p(z,t) c(z,t)p(z,t) dz - P(t)\mu_P
\end{align}
The concentration of prey and predators is naturally given by a product of probability densities $\phi_i,~i\in \{c,p\}$, describing their location and the total amount of predators and prey.
\begin{align}
  \label{eq:prob_dens}
	c(z,t) &= C(t)\phi_c(t, z) \\
	p(z,t) &= P(t)\phi_p(t, z)
\end{align}
Incorporating \Cref{eq:prob_dens} in \Cref{eq:population_growth}, we arrive at equations for the population dynamics governed by probability densities:
\begin{align}
  \label{eq:population_growth_prob_dens}
	\dot{r} &= r(z,t)\pa{1-\frac{r(z,t)}{r_{max}(z)}} - \beta_c(z,t)\phi_c(z,t)C(t) r(z,t)  + k \partial_x^2 r(z,t)\\
	\dot{C} &= C(t)\left ( \int_0^{z_0} \varepsilon \beta_c(z,t)\phi_c(z,t)r(z,t) dz- \int_0^{z_0} \beta_p(z,t) \phi_c(z,t) p(z,t) dz - \mu_C \right ) \\
	\dot{P} &= P(t) \left ( \int_0^{z_0} \varepsilon \beta_p(z,t) c(z,t)\phi_p(z,t) dz - \mu_P \right )
\end{align}


\subsubsection*{Fitness proxies and optimal strategies}
%Fitness proxies
%Add introduction to fitness.. Potentially split into growth and loss terms, potentially do this further up.


The instantaneous fitness of an individual forage fish $(F_p)$ or copepod $(F_c)$ is given by its growth rate at that instant. As fitness is an individual measure, we arrive at the fitness by dividing the population growth rate \Cref{eq:population_growth_prob_dens} by the total population.

\begin{align}
	F_c(\phi_c, \phi_p) &= \int_0^{z_0} \varepsilon \beta_c(z,t)\phi_c(z,t)r(z,t) dz\\ &- P(t)\int_0^{z_0} \beta_p(z,t) \phi_c(z,t) \phi_p(z,t)dz \\
	F_p(\phi_c, \phi_p) &=  C(t) \int_0^{z_0} \varepsilon \beta_p(z,t)\phi_c(z,t)\phi_p(z,t) dz
\end{align}
Optimal strategies:

At any instant, an organism seeks to find the strategy that maximizes its fitness. A strategy in our case is a probability distribution in the water column. The optimal strategy $\phi_c^*$ of a consumer depends on the strategy of the predators, and likewise for $\phi_p^*$ for the predators. Denoting  the probability distributions on $[0,z_0]$ by $P(0,z_0)$, this can be expressed as:
\begin{align}
	\phi_c^*(z,t)(\phi_p) &= \argmax_{\phi_c \in P(0,z_0)}  \int_0^{z_0} \varepsilon \beta_c(z,t)\phi_c(z,t)r(z,t) dz \\ &- P(t)\int_0^{z_0} \beta_p(z,t) \phi_c(z,t) \phi_p(z,t)dz  \\
	\phi_p^*(z,t)(\phi_c) &= \argmax_{\phi_p \in P(0,z_0)}C(t) \int_0^{z_0} \varepsilon \beta_p(z,t)\phi_c(z,t)\phi_p(z,t) dz
\end{align}

Consumers and predators maximize their fitness simultaneously, leading to a \emph{Nash Equilibrium}, where neither can gain anythin from diverging from their strategy. The Nash equilibrium of the instantaneous game is:
\begin{align}
  \label{eq:nash_equilibria}
	\phi_c^{*,NE} &=  \argmax_{\phi_c \in P(0,z_0)}  \int_0^{z_0} \varepsilon \beta_c(z,t)\phi_c(z,t)r(z,t) dz \\ &- P(t)\int_0^{z_0} \beta_p(z,t) \phi_c(z,t) \phi_p^{*,NE}(z,t) dz \\
	\phi_p^{*,NE} &=  \argmax_{\phi_p \in P(0,z_0)}C(t) \int_0^{z_0} \varepsilon \beta_p(z,t)\phi_c^{*,NE} \phi_p(z,t) dz
\end{align}

\subsubsection*{Noisy strategies} %	Y \sim \mathcal{N}(0, \sigma^2) \\
Our model incorporates that fish are not necessarily perfectly rational, but have \textbf{bounded rationality} by letting the strategy depend on the parameter $\sigma$ as well, with $\sigma=0$ being completely rational and $\sigma = \infty$ completely irrational. Rather than choosing a precise location, an individual can choose where it diffuses around. As fish cannot swim out of the top of the ocean, nor through the bottom, we end with the partial differential equation:
\begin{align}
  \label{eq:density_PDE}
	&\partial_\sigma \phi_i = \partial_z^2 \phi_i \\
	&\partial_z \phi_i \mid_{z=0} = 0 \\
  &\partial_z \phi_i \mid_{z = z_0} = 0
\end{align}
Leting $\phi$ denote the density of a standard normal distribution, \Cref{eq:density_PDE} has the solution:
\begin{align}
  \phi_i(x_0,z,\sigma=0) &= \delta(z-x_0) \\
  \phi_i(x_0,z,\sigma) &=\kappa(x_0) \frac{1}{\sqrt{2\sigma}} \pa{\phi\pa{\frac{z-x_0}{\sqrt{2\sigma}}} + \phi\pa{\frac{x_0 - z}{\sqrt{2\sigma}}} }
\end{align}
where $\kappa(x_0)$ is a normalization parameter to ensure that $f_Y$ is a probability density. The fundamental solution, or Greens function, is thus
\begin{align}
  \label{eq:Greens_function}
  f_Y =\kappa \frac{1}{\sqrt{2\sigma}} \pa{\phi\pa{\frac{z}{\sqrt{2\sigma}}} + \phi\pa{\frac{- z}{\sqrt{2\sigma}}} }
\end{align}


If a consumer has an initial strategy defined by a random variable $X_c$ with density $f_{X_c}$, to find the final strategy $\phi_c$ we need to solve \Cref{eq:density_PDE} with initial value $f_{X_c}$ and rationality $\sigma$. This is found by convolution with the fundamental solution, \Cref{eq:fundamental_solution}:
\begin{align}
  \label{eq:realized_distribution}
	\phi_c = f_X * f_Y(\sigma)
\end{align}
%Remark that if $Y$ is a random variable following a truncated normal distribution on $[0,z_0]$ with density $f_Y$, then $\phi_c$ is exactly the density of the random variable $X_{c}+Y$.


Having introduced noise to the strategies of consumers and predators, we can find the Nash equilibrium of their optimal distributions without noise. We find the Nash pair by inserting \Cref{eq:realized_distribution} in \Cref{eq:nash_equilibria} and optimizing over $f_{X_i},~i\in \{c, p\}$.
\begin{align}
	f_{X_c}^{*,NE} &=  \argmax_{f_{X_c} \in P(0,z_0)}  \int_0^{z_0} \varepsilon \beta_c(z,t)(f_{X_c} * f_Y) r(z,t) dz \\ & -  P(t)\int_0^{z_0} \beta_p(z,t) (f_{X_c} * f_Y)(f_{X_p}^{*,NE} * f_Y) dz \\
	f_{X_p}^{*,NE} &=  \argmax_{f_{X_p} \in P(0,z_0)}C(t) \int_0^{z_0} \varepsilon \beta_p(z,t)(f_{X_c}^{*,NE}  * f_Y)(f_{X_p}* f_Y) dz
\end{align}

The realized distributions are found by convolution with $f_Y$, \Cref{eq:realized_distribution} as
\begin{align}
  \phi_c^{*,NE} &= f_{X_c}^{*,NE} * f_Y \\
  \phi_p^{*,NE} &= f_{X_p}^{*,NE} * f_Y
\end{align}

\subsubsection*{Spatial discretization}
We discretize the interval $[0,z_0]$ with a Gauss-Lobatto grid, \citep{kopriva2009implementing}. Working on a Gauss-Lobatto grid, we are working on a grid naturally accomodated to Legendre polynomials, as the nodes correspond to zeros of Legendre polynomials. This allows the use of spectral methods for integration and differentation, which are fully implicit. Spectral methods are particularly well-suited to working with smooth functions, as the accuracy of integration and procedures on smooth-functions improves faster than any polynomial as a function of the number of grid points, \citep{kopriva2009implementing}.
We approximate pure strategy of being in a point $z_i$  by a hat-function $h_i$, zero everywhere apart from $z_i$. Due to the non-constant interval size we need to find constants $\alpha_i$ so $h_i$ integrates to 1.
\begin{align}
	&\alpha_i \int_{z_i}^{z_{i+1}} \tilde{h}_i dz = 1 \\
	&\tilde{h}_i(z_{i-1}) = 0,~ \tilde{h}_i(z_{i+1}) = 0
  &h_i(z)=\alpha_i \tilde{h}_i(z)
\end{align}
Working on a grid with $N$ points, a strategy chosen by a consumer or predator then becomes a linear combination of hat-functions,
\begin{align}
  &\phi_{i} = \sum_{j<N} a_{j,i} h_j, \quad i\in \{c,p\} \\
  &\sum_{j<N} a_{j,i} = 1 \quad i\in \{c,p\}
\end{align}
. The strategy of a player is fully determined by the $a_i$'s, and using spectral integration we can easily determine integrals of the type $\int_0^{z_0} \phi_c \phi_p dz$

When considering non-optimal actors, we need to implement the convolution with $f_Y$, \Cref{realized_distribution}. This is by initially calculating a convolution matrix $C$, \citep{}, which incorporates the normalization constants $\kappa(x_0)$. Using $C$, we can calculate a convolution of a function $f$ with $f_Y$, \Cref{eq:realized_distribution} by taking the matrix-vector product with $C$. Thereby we get a new set of pure strategies, $\hat{h}_i = h_i * f_Y$.
An added benefit of incorporating bounded rationality then becomes that our strategy profiles are guaranteed to be smooth, decreasing the number of points needed for exact evaluation of the integrals.


\subsubsection*{Finding the nash equilibrium}
By discretizing space, we have reduced an uncountable strategy set to a more manageable finite amount, with pure strategies $h_i$, or $\hat{h}_i$. For brevity, we simply lump them together as $e_i$. The gain of a consumer playing strategy $e_i$ against a predator playing strategy $e_j$ can be determined as $F_c(e_i,e_j)$, and similary for a predator. This allows us to write up payoff matrices $E_c, E_p$, with entry $(i,j)$ determined through $F_k(e_i,e_j), k \in \{c, p\}$.  Both payoff functions are bi-linear in the strategies, so our discretization has reduced the problem to a bimatrix game. A bimatrix game is a special case of a polymatrix game, where $n$ players play against each other in pairwise games and the total payoff is given by the sum of the payoffs across the pairwise interactions. %, which can be done efficiently via. our spectral discretization.



Polymatrix games can be solved by passing to an equivalent linear complementarity problem,  \citep{miller1991copositive}. In our case we have a bimatrix game, but the approach is the same for games with more players, and has been implemented in the code. The first step is to introduce the total payoff matrix:
\begin{align}
	R_{init} = \begin{bmatrix} 0 & E_c \\ E_p \end{bmatrix}
\end{align}
As all entries in $R_{init}$ do not have the same sign, $R_{init}$ is not copositive. We fix this by defining $R=R_{init}-max(R_{init})$.
Applying the results of \citep{miller1991copositive}, to find the Nash equilibrium we need to solve the problem:
\begin{align}
	(Hz+q) = w & \\
	\ip{z}{w} = 0 & \\
	z\geq 0, w\geq 0
\end{align}
where
\begin{align}
	A &= \begin{bmatrix} -1 &-1 & \dots & -1 \\  -1 &-1 & \dots & -1 \end{bmatrix} \\
	q &= \begin{pmatrix} 0 &\dots & 0 & -1 & -1 \end{pmatrix}   \\
	H &= \begin{bmatrix} -R & -A^T \\ A & 0 \end{bmatrix}
\end{align}

This was done through two different methods. The interior-point method as implemented in IPOPT, \citep{wachter2006implementation}, called via. the auto-differentation software CasADi \citep{Andersson2019}, and Lemkes Algorithm implemented in the Numerics package in Siconos, \citep{acary2019introduction}.

\subsubsection*{Time evolution}

We solve the time-evolution of the predator and prey populations using a semi-implicit euler scheme. At each step we find the Nash equilibrium based on the last state, and evolve the populations accordingly. The time-evolution of the resource is solved by the method of exponential time differencing, using a first-order difference, \citep{hochbruck2010exponential}. That is, we write up the exact formula for the solution of $r(z,t+\Delta t)$ based on $r(z,t)$
\begin{align}
  r(z,t_{i+1}) &= \exp(\Delta t k \part_x^2) r(z,t_{i}) \\
  & +\int_{t_i}^{t_{i}+\Delta t} \exp(t' k \part_x^2) r(z,t')\pa{1-\frac{r(z,t')}{r_{max}(z)}} \\
  & - \beta_c(z,t')\phi_c(z,t')C(t') r(z,t') dt'
\end{align}
We recognize $\exp(\Delta t k \part_x^2)$ as acting by convolution with the Greens function $G$ of the heat equation, previously seen in the guise of $f_Y$, \Cref{eq:Greens_function}, with $G(k,\Delta t)=f_Y(k\Delta t)$.
Doing a first order-approximation, defining
\begin{align}
  J(z,t_{i}+\Delta t) &= r(z,t_i) \\
  &+ \left ( r(z,t_i)\pa{1-\frac{r(z,t_i)}{r_{max}(z)}} \right . \\
  & \left . - \beta_c(z,t_{i+1})\phi_c(z,t_{i+1})C(t_{i+1}) r(z,t_i) \right ) \Delta t
\end{align}
We end with the approximation:
\begin{align}
	r(z,t_i + \Delta) \approx G(k, \Delta t)*J(z,t_{i}+\Delta t)
\end{align}
The choice of using an exponential integrator ensures smoothness of the solution is preserved numerically, and in general the method of exponential integrators is well-suited for stiff problems, \citep{hochbruck2010exponential}.

\subsubsection*{Model parametrization}
Following \citep{yodzis1991}, and \citep{kha_2019}, we parametrize our model in a metabolically scaled manner following Kleibers law, \citep{kleiber}.



\begin{tabular}{l | l | l}
  Precipitable water & $w_a$ & 1 $g \cdot m^{-3}$\\
  Aeorosol optical depth & $aod$ & 0.1 \\
  Light decay & $k$ & $0.05 m^{-1}$\\
  Ocean depth & $z_0$ & $170 m$ \\
  Minimal attack rate & $\beta_0$ & $5 \cdot 10^{-3} m^{3} year^{-1}$ \\
  Consumer mass & $m_c$ & 0.05 $g$ \\
  Predator mass & $m_p$ & 20 $g$ \\
  Consumer clearance rate & $\beta_c$ & 32 $m^{3} year^{-1}$ \\
  Predator clearance rate & $\beta_p$ & 2837 $m^3 year^{-1}$ \\
  Phytoplankton growth & $\lambda$ & 300 $year^{-1}$ \\
  Phytoplankton max & $r_{max}$ & $10\mathcal{N}(0,3)$ \\
  Irrationality & $\sigma$ & 160 $m^2 year^{-1}$ \\
  Diffusion rate & k & 500 $m^{2} year^{-1}$
\end{tabular}
